\label{sec:scalability}

% CCC: Moved scalability figure to evaluation.tex to place on proper page

To measure the scalability cost of running \dthreads{}, we ran our
benchmark suite (excluding \texttt{canneal}) on the same machine with
eight cores, four corse, and just two cores enabled.  Whenever possible
without source code modifications, the number of threads was matched
to the number of CPUs enabled.  We have found that \dthreads{} scales
at least as well as \pthreads{} for 9 of 13 benchmarks, and scales as
well or better than CoreDet for all but one benchmark
where \dthreads{} outperforms CoreDet by $3.5\times$.  Detailed results of this
experiment are presented in Figure~\ref{fig:scalability} and discussed
below.

The \texttt{canneal} benchmark was excluded from the scalability experiment because it matches the workload to the number of threads, making
the comparison between different numbers of threads invalid.  \dthreads{}
hurts scalability relative to \pthreads{} for the \texttt{kmeans}, \texttt{word\_count}, \texttt{dedup}, and \texttt{streamcluster} benchmarks,
although only marginally in most cases.  In all of these cases, \dthreads{} scales better than CoreDet.

\dthreads{} is able to match the scalability of \pthreads{} for three
benchmarks: \texttt{matrix\_multiply}, \texttt{pca},
and \texttt{blackscholes}.  With \dthreads{}, scalability
actually \emph{improves} over \pthreads{} for 6 out of 13
benchmarks.  This is because \dthreads{} prevents false sharing, avoiding 
unnecessary cache invalidations that normally hurt scalability.

% \textbf{tongping: Should we cite "sheriff" here?}
