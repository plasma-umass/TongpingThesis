Different with CoreDet, \dthreads{} are not scheduled based on instruction quatum 
but on synchronization points.  

There are two advantages by doing this:
First, there is no read barriers and no write barriers for memory accesses. 
CoreDet doesn't allow reads/writes on one memory unit owned by others and writes on one shared memory unit.
Those operations should wait until the serial phase, which can affect performance greatly. 
While threads are treated as processes in \dthreads{} and memory are read-shared across different process. 
Read operations can happen directly on shared memory without barrier and write operations only invoke one 
trap and one commit, thus it improves performance for most of benchmarks. 

Second, \dthreads{} are scheduled based on more natural synchronization points(not instruction quatum).
Thus \dthreads{} can provide stable schedule if the synchronization are not changed accross different runs, but 
no need to memoize schedules~\cite{stable-deterministic}.
Also, modification without the change of synchronization won't change the schedule order, which can be helpful to debug 
and reason about the program.

%Emery: If there is no synchronization, \dthreads{} can work on its private copy safely.
%Advantages over prior work; performance (no read barriers and no
%per-access write barriers (just once per dirtied page)). especially
%robustness.

% \subsection{Disadvantage of \dthreads{}}
