\label{sec:discussion}

This section analyzes some key limitations of \dthreads{} that
restrict its ability to run certain programs, limit the extent of
determinism it can guarantee, or potentially affect performance.


\textbf{Unsupported programs: }
\dthreads{} currently does not support programs with ad hoc
synchronizations, such as those that use atomic operations implemented in assembly.  However, the upcoming C++0X standard includes a library interface for atomic operations~\cite[pp. 1107--1128]{c++0xstandarddraft}, and a future version of \dthreads{} could correctly implement these by intercepting
these library calls and treating them as synchronization points. While
ad hoc synchronization is a common practice, it is also a notorious
source of bugs; Xiong et al.\ show that 22--67\% of the uses of ad hoc
synchronization lead to bugs or severe performance issues~\cite{ad-hoc-considered-harmful}.

\dthreads{} also currently does not write-share the stack
across threads, so that updates made by a thread to a stack variable
would not be reflected back to the parent, which could cause a program
to fail. Passing stack variables to a thread for modification is
extremely error-prone and generally deprecated, making this a rare
coding practice.

\textbf{External determinism: }
While \dthreads{} provides internal determinism, it does not
guarantee determinism when a program's behavior depends on external
sources of non-determinism, such as system time or I/O
events. Incorporation of \dthreads{} in the dOS framework, an OS
proposal that enforces system-level determinism, would provide full
deterministic execution, although this remains future
work~\cite{deterministic-process-groups}.

\textbf{Runtime performance: }
Section~\ref{sec:evaluation} shows that \dthreads{} can provide high
performance for a number of applications; in fact, for the majority of
the benchmarks examined, \dthreads{} matches or even exceeds the
performance of \pthreads{}. However, \dthreads{} could occasionally
degrade performance, sometimes substantially. One way it could do so
would be to exhibit an intensive use of locks (that is, acquiring and
releasing locks at high frequency), which are much more expensive
in \dthreads{} than in \pthreads{}. However, because of its determinism guarantees, \dthreads{} could allow programmers to greatly reduce their use of locks, and thus improve performance. Other application characteristics, also explored in Section~\ref{sec:performance}, can also impair performance
with \dthreads{}.


%  Since Surprise locking inside libraries. Not a limitation \emph{per
%  se} but definitely an issue that could surprise programmers.

% Draft can be downloaded from http://www.open-std.org/jtc1/sc22/wg21/docs/papers/2010/n3126.pdf.
%Fine once they are library calls, as they are in gcc and in the upcoming C++0X standard (cite!), since then we can intercept them.

\textbf{Memory consumption: }
Finally, because \dthreads{} creates private, per-process copies of
modified pages between commits, it can increase a program's memory
footprint by the number of modified pages between synchronization
points. This increased footprint does not seem to be a problem in
practice, both because the number of modified pages is generally far
smaller than the number of pages read, and because it is transitory:
all private pages are relinquished to the operating system
(via \texttt{madvise()}) at the end of every commit operation.

%Increased memory footprint (linear in the number of dirtied (modified) pages).


