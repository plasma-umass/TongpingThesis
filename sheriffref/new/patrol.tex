\label{sec:patrol}

While \sheriffdetect{} can be effective at locating the sources of
false sharing, it is sometimes difficult or impossible for programmers
to remove the false sharing that \sheriffdetect{} can reveal. For
instance, padding data structures to eliminate false sharing within an
object or different elements of an array can cause excessive memory
consumption or degrade cache utilization~\cite{zhao:vee:2011}. Time
constraints may prevent programmers from investing in other solutions,
or the source code may simply be unavailable. \sheriffprotect{}, our
second tool developed with the \sheriff{} framework, can take
unaltered C/C++ binaries and eliminate the performance degradation
caused by false sharing.

To accomplish its goals, \sheriffprotect{} relies on a key insight
due initially to DuBois et al.\
~\cite{Dubois:1991:DCE:125826.125941}: \emph{delaying updates avoids
  false sharing.}  Delaying one thread's updates
so that they do not cause invalidations to other threads 
eliminates the performance impact of false sharing. Consider the case when
two threads are updating two falsely-shared objects A and B. If one
thread's accesses to A were delayed so that they preceded all of the
other thread's accesses to B, then the false sharing would cause no
invalidations and hence have no performance impact.

In effect, this is exactly what \sheriff{} does already when all pages
are mapped private. By using processes instead of threads, all updates
between synchronization points are applied to different physical addresses
belonging to the different ``threads''. In this way, \sheriff{} itself
prevents false sharing by not updating the same physical cache
lines.

However, using \sheriff{} with all pages mapped private would be impractical as a runtime
replacement for \pthreads{}, because it would impose excessive overhead.
This overhead arises due to 
protection and copying costs that would counteract the benefit of
preventing false sharing.

For example, when a thread updates a large number of pages between
transactions, \sheriff{} must commit all local changes to the shared
mapping at the end of every transaction, \emph{even in the absence of
  sharing}. The associated protection and copying overhead can
dramatically degrade performance. The same problem arises when transactions are
short (e.g., when there are frequent lock acquisitions), because the
cost of protection and page faults cannot be amortized by the
protection-free part of the transaction.

\sheriffprotect{} implements two key optimizations that improve performance by directly addressing these issues.

\paragraph{Focus on smaller objects:}
  \sheriffprotect{} focuses its false sharing prevention exclusively
  on small objects (those less than 1024 bytes in size). All large
  objects are mapped shared and are never protected. 

  First, we expect small objects to be a likely source of false
  sharing because they fit on a cache line. False sharing in large
  objects like arrays is also possible, but we expect the amount of
  false sharing relative to the size of the object to be far lower
  than with small objects (an intuition that \sheriffdetect{}
  confirms, at least across our benchmark suite). For such objects,
  the cost of protection would outweigh the advantages of
  preventing false sharing.

  Second, the total amount of memory consumed by small objects tends
  to be less than that consumed by large objects. Because the cost of
  protecting and committing changes is proportional to the volume of
  updates, \sheriffprotect{} limits protection to small objects,
  reducing overhead while capturing the benefit of false sharing
  prevention where it matters most.

\begin{table*}[!t]
\centering
\begin{tabular}{l|c|c|c}
\hline
{\bf \small Microbenchmark} & {\bf \small Performance-Sensitive /} & {\bf \small \sheriffdetect{} } & {\bf \small PTU } \\
 & {\bf \small Actual False Sharing?} & & \\
\hline

\small {\em False sharing (adjacent objects)} & Yes & $\checkmark$ & $\checkmark$\\
\small {\em Pseudo sharing (array elements)} & Yes & $\checkmark$ & $\checkmark$\\
\hline
\small {\em True sharing} & No &  & \\
\small {\em Non-interleaved false sharing} & No  & & $\times$\\
\small {\em Heap reuse (no sharing)} & No & & $\times$\\
\hline
\end{tabular}
\caption{False sharing detection results using PTU and \sheriffdetect{}. 
\sheriffdetect{}
correctly reports only actual false sharing instances, and only those with a
performance impact; $\checkmark$ denotes a correct report, and
$\times$ indicates a spurious report (a false positive).
\label{table:microbenchmarks}}
\end{table*}


\paragraph{Adaptive false sharing prevention:} 
  Short transactions do not give \sheriff{} the chance to amortize its
  protection overheads. To address this, \sheriffprotect{} employs a simple adaptive
  mechanism.  \sheriffprotect{} tracks the
  length of each transaction and uses exponential weighted averaging
  ($\alpha = 0.9$) to calculate the average transaction length.  Once
  the average transaction length is shorter than an established
  threshold, \sheriffprotect{} switches to using shared mappings for
  all memory and does no further page protection. As long as
  transactions remain too short for \sheriffprotect{} to have any
  benefit, all of its overhead-producing mechanisms remain switched
  off. If the average transaction length rises back above the
  threshold, \sheriffprotect{} re-establishes private mappings and
  page protection.

