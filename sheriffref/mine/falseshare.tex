\label{sec:falseshare}
From above section, we already know that how to design a runtime system to simulate the running 
of multi-threaded program. 

In this section, we are going to talk how to indentify false sharing problems by recording memory writing using
the runtime system.
This section are trying to answer the following questions:

\begin{itemize}
\item How to capture the memory writes from different process?
\item How to capture the continuous memory writes?
% - sampling mechanism - "temporary twin" and "original twin". 
\item How to capture the interleaving cache invalidation? 
% Use a global array, updating timely when modification is detected. 
\item How to identify objects inside one cache line? 
%Attach the callsite information to capture the allocate sites for heap objects. 
\item How to differentiate true sharing and false sharing?
%detect the combination? An array to get word version and threads working on that. We can detect those fields inside one object causing the problem too.
\item How to report one false sharing problem?
% In the end of program, we traverse the whole global array.
\end{itemize}

\subsection{Capture of Memory Writes}
\label{falseshare:memorywrites}
Process can provide a strong isolation of one thread's running from other threads' running. 
In each transaction, Sheriff runs one thread in a atomical, consistent and isolated way and
won't commit those changes in one transaction until the end of one transaction.
In the end of each transaction, Sheriff can compare ``twin'' page and ``working'' page word by word to find 
those modifications on each dirty page. When the word of ``working'' page is different from that of 
corresponding ``twin'' page, this word is thought to be modified by current thread in current transaction. 
It is reasonable to reach this conclusion since Sheriff can guarantee that originally the content of ``twin'' page 
is the same as that of ``working'' page by forcing a COW explicitely (see ~\ref{simulation:execution}).
\begin{comment}
It is true that the writing of ``A-B-A''  can be missed by simply comparison,
but we believe that ``A-B-A'' writing in one transaction
is not frequent and won't bring any correctness problem.
We don't want to put too much focus on this point.
\end{comment}

Since we can capture the memory writes on every transaction and one thread's running is consisted of multiple transactions, 
we can capture the memory writes from different threads.

\subsection{Capture of Continuous Writes}
\label{detection:sampling}
We already know from above section that Sheriff can capture memory writes on one transaction. 
But it is not good enough when the transaction length is too long (some extreme case can be the whole thread). 
Actually, one serious false sharing problem (\texttt{linear\_regression} benchmark, see Section~\ref{sec:evaluation}) 
which affect the performance 10X can be omitted since there is only one transaction for one thread, 
without any synchronization inside. 

Sheriff use a sampling mechanism to avoid this problem. Sampling is 
to select some of observations in order to acquire some knowledge about the whole.
Although sampling cannot give complete information about memory writes on one transaction,
sampling can be used to capture more writes. More fine sampling can help to find more writes by one transaction.
There is a balance between choosing finer sampling period and performance issue here. 
Sheriff now choose 10 microseconds as a basic interval to do sampling. 

In order to capture continuous writes, Sheriff introduce one ``temporary twin'' page for every shared dirty page
 (see Fig.\ref{fig:overview}). 
Handling of those ``temporary twin'' pages are slightly diferent with those ``original twin'' pages.
First, they are created in the sampling timer handler when one page is found to be shared by multiple threads. 
There is no use to create ``temporary twin'' for those pages only accessed by one thread. We are using a global array to
record users for one page. 
Second, ``temporary twin'' pages are keeping updated to ``working'' version in every timer handler 
in order to capture future writes on the same page.
 
\subsection{Capture of Cache Invalidations}
\label{detection:invalidation}
Just as we talked in Section~\ref{overview:target}, only numerous interleaving writes can bring performance problem. 
Sheriff tries to capture the interleaving writes across different threads in order to capture 
cache invalidations. 

In order to capture interleaving writes on caches, Sheriff introduces 
virtual cache line status words (Fig.~\ref{fig:overview}). 
``virtual'' is used here to differentiate with ``physical'' cache line. 
For every virtual address range (same size with physical cache line) under protection, Sheriff assigns one status word. 
One status word has two fields, the first field points last thread to write on this cache line, 
the second field is used to record times of invalidates (version) on one cache line. 
Every time when one different thread are detected to write on this cache line, Sheriff update both the thread id
to be the new thread and version number. 
In actual implementation, Sheriff introduce two different arrays to avoid using lock. Corresponding code can be seen
in Figure~\ref{fig:capturecacheinvalidation}.
CacheInvalidation array is used to capture those interleaving cache invalidation for all cache lines in protected memory. 
Every cache line have one corresponding counter to indicate the interleaving of cache invalidation for this cache line. 
LastThreadModifyCache array is used to record last thread id to write on its cache line. 

The pseudo code to capture the interleaving cache invalidation is listed in Figure~\ref{fig:capturecacheinvalidation}:
\begin{figure*}[!t]
\begin{lstlisting}
void recordCacheInvalidates(int cacheNo) {
    int myTid = getpid();
    int lastTid;

    // Try to check last thread to modify this cache.
    lastTid = atomic_exchange(&LastThreadModifyCache[cacheNo], myTid);
    if(lastTid != myTid) {
       // Increment cache invalidation only when current thread is different.
       atomic_increment(&cacheInvalidation[cacheNo]);
     }
}
\end{lstlisting}
\caption{Record the cache invalidation atomically.\label{fig:capturecacheinvalidation}}
\end{figure*}

\subsection{Indentify Objects inside Cache Line}
\label{detection:object}
For global object, Sheriff don't need to do anything since debug information can provide
object's information.
Sheriff attaches the call site in the header of each heap object when allocation to indentify objects.
Callsite information can provide objects' request allocation, which is useful for programmer
to fix the false sharing problem (see case study in Section~\ref{evaluation:comparison}).
It is one important feature to differentiate Sheriff from previous tools.
Previous tools using binary instrumentation or hardware performance counter cannot control
the memory allocation, so they cannot provide the callsite information about one object.
Sheriff is a runtime system which intercepts all heap allocations so that Sheriff can tell programmer
about cache line's object information.

Remember that Sheriff have two arrays to capture the cache interleaving invalidation
(see Section~\ref{detection:invalidation}), it is necessary to cleanup those invalid counting
when one object is de-allocated. It is important to avoid the false positives caused by uncorrectly aggregate 
counting when one address is re-used for other objects. 

\subsection{Avoidance of False Positives}
\label{detection:avoidfalsepositive}
To avoid false positives, 
Sheriff introduces another global array to record  
threads writing on each word and version numbers of each word.
Threads writing on each word can tell whether one cache line is false sharing or true sharing. 
Version number on each word can avoid to report those objects which don't contribute much on cache invalidations, 
when there are multiple objects in the same cache line.
In order to save space, Sheriff use one word's higher 16 bit to store the thread id on one word
and use the lower 16 bit to store version number of this word. 
When one word is detected to be modified by more than two threads, we marked specially
on its thread id field.

\subsection{Reporting False Sharing Objects}
In the end of program, Sheriff reports those objects causing false sharing problems. 
Since Sheriff introduce a global array (CacheInvalidationArray) to record those 
cache invalidation (see Section~\ref{detection:invalidation}), Sheriff checks 
CacheInvalidationArray for cache lines with invalidation times larger than one water level. 
After one cache line is found, corresponding invalidation times and offset of this cache line 
will be added into a global link linst sorted by invalidation times. 
Later we can rank the false sharing objects by invalidation times they caused. 

After the traverse of all cache lines, Sheriff tries to get objects information 
for all cache lines in the link list. 
Sheriff uses magic value added in the allocation to differentiate the start of one object. 
Also, the object size information can help to identify the start of one object. 
After finding out those objects inside one cache line, Sheriff should look into the 
array listed in Section~\ref{detection:avoidfalsepositive} to avoid false positives. 

%%%%%%%%%%%%%%%%%%%%%%%%%%%%%%%%%%%%%%%%%%%%%%%%%%%%%%%%%%%%%%%%%%%%%%%%%%%%%
%%%%%%%%%%%%%%%% Where to specify those procedure of timer handler????? LTP
%%%%%%%%%%%%%%%%%%%%%%%%%%%%%%%%%%%%%%%%%%%%%%%%%%%%%%%%%%%%%%%%%%%%%%%
