\label{chapter:preditor}

This chapter presents \Predator{}, which improves the effectivenss of false sharing detection. \SheriffDetect{} reports false sharing accurately and precisely with only $20\%$ performance overhead. However, it can only detect the write-write type of false sharing for those programs using \pthreads{} library. \SheriffDetect{} can also break programs communicating across different threads with stack variables or self-defined synchronizations. These shortcomings greatly limit \SheriffDetect{}'s usage on real-world applications.  

In contrast to \SheriffDetect{}, \Predator{} detects all types of false sharing and has no limitations on applications. \Predator{} has been utilized to find actual false sharing in real applications, including \texttt{MySQL} and the \texttt{Boost} library.

In addition, \SheriffDetect{} and other systems share one key limitation: they can only report \emph{observed} cases of false sharing. As Nanavati et al.\ point out, false sharing is sensitive to where objects are placed in cache lines and so can be affected by a wide range of factors~\cite{OSdetection}. For example, using the gcc compiler \emph{accidentally} eliminates false sharing in the Phoenix linear\_regression benchmark at certain optimization levels, while LLVM does not do so at any optimization level.  A slightly different memory allocation sequence (or different memory allocator) can reveal or hide false sharing, depending on where objects end up in memory; using a different hardware platform with different addressing or cache line sizes can have the same effect. All of this means that existing tools cannot root out potentially devastating cases of false sharing that could arise with different inputs, in different execution environments, and on different hardware platforms.

\Predator{} is the first system that can \emph{predict} potential false sharing that does not manifest in an execution but may appear and greatly degrade the performance of programs in a slightly different
environment. Predictive false sharing generalizes from a single execution to identify potential false sharing instances that are within one cache line of each other, which could be exposed by slight changes in object placement and alignment. It also can predict false sharing in hardware platforms with larger cache line sizes by tracking accesses within \emph{virtual cache lines} that span multiple physical lines. Predictive false sharing detection thus overcomes a key limitation of previous detection tools.

\section{False Sharing Detection}
\label{sec:detectfalseshare}

This section first describes the basic idea of detecting false sharing. 

\subsection{Basic Idea}
\label{sec:detectionidea}

There is a false sharing problem when two threads simultaneously access independent data in the same cache line. False sharing does not necessarily cause performance problems. It can greatly degrade performance only when those accesses, caused by threads running on different cores with separate cache, actually cause a big number of cache invalidations. This is our \textbf{basic observation}. 

However, it is hard to know where a thread is located in the user space. Moreover, it is unnecessary to know this relationship since the relationship between threads and cores can be changed from one execution to the other. Thus, we make the following \textbf{assumptions}. First, we conservatively assume that all threads are running on different cores, with separate caches. Thus we can report the worst-case results caused by false sharing problems. In order to avoid expensive cache simulation, which can be affected by cache hierarchy, cache capacity and cache eviction rule, we further assume that the data is never evicted from its private cache by cache eviction, with infinite cache. This second assumption allows us to predict the cache invalidations based on memory accesses only, no need to think about cache eviction.   

Based on our basic observation, detecting false sharing problems turns into locating cache lines with a big number of cache invalidations. Cache invalidations are invoked by cache coherence protocol, the protocol used to ensure the coherence of data in the multiple-processor system. Based on our assumptions, there is a cache invalidation if a processor writes a cache line after another thread's access on the same cache line. Because the last thread accessing this cache line creates a copy of the same cache line on its running core's private cache (first assumption) and  holds this copy(second assumption), this write operation definitely cause a cache invalidation before or after this write operation, depending on different cache coherence protocols. 

\begin{figure}[!t]
\centering
\includegraphics[width=5in]{fig/cachelinestatuswords}
\caption{
To detect false sharing, each cache line of the globals and heap maintains a cache line status word, which is updated on each tracked memory access. \label{fig:cachelinestatusword}}
\end{figure}


To locate cache lines with a big number of cache invalidation, we maintain a cache line status word for each cache line in the globals and heap, shown in Figure~\ref{fig:cachelinestatusword}. We share the similar mechanism as another concurrent work of Zhao et.al. ~\cite{qinzhao}. However, the detailed implementation is totally different. Zhao et.al. utilize the detailed ownership bitmap and last thread bitmap to precisely track the cache invalidation, which can even track how many cache invalidations may happen in a write operation. However, their design can not easily scale to more than 32 threads, requiring more memory overhead caused by more bits and more checking performance overhead. Also, their approach miss one important factor - how many cache invalidations happening on a specific cache line. Without this information, it is impossible to pinpoint false sharing problems that can cause performance problems.   
Our approach overcomes all these shortcomings, by only tracking the last thread index and the number of cache invalidations. Thus, we can rank the seriousness of false sharing problems based on the number of cache invalidations. 

\subsubsection{Accurate Detection}
\label{sec:accuratedetect}

Accurate detection implies that we only report those false sharing problems that can cause performance problems. 

First, we only report those problems that can cause performance problems. This has been resolved by only reporting false sharing with a big number of cache invalidations. 

Second, this also implies that we should be able to differentiate false sharing from true sharing. As we all know, true sharing also can cause cache invalidations, which is the essence to have coherence protocol. In order to differentiate false sharing with true sharing, we further  track word-level access information for those cache lines involved in false sharing: how many reads or writes to each word by which thread. When a word is accessed by multiple threads, we mark the origin of this word as a shared access and do not track threads for further accesses to it. This information lets us accurately distinguish false sharing from true sharing in the reporting phase. It also helps diagnose where actual false sharing occurs when there are multiple fields or multiple objects in the same cache line, as this can greatly reduce the manual effort required to fix the false sharing problems.
  
Third, we should avoid pseudo false sharing (false positives) caused by memory reuses.  We update recording information at memory de-allocations for those objects without false sharing problems; heap objects involved in false sharing are never reused so that they can be reported in the end or on demand. 

\subsubsection{Precise Detection}
\label{sec:precisedetect}

Precise detection implies that we can precisely point out where the problem is. Thus, programmers can leverage on that to identify and correct false sharing. 

We identify globals directly by using debug information that associates the address with the name of the global. In order to precisely report the origins of heap objects with false sharing problems, we collect callsite information for each heap object by intercepting memory allocations and deallocations, and report to users about the origins of false sharing objects. 

Also, we present the word-level accesses information to users so that the exact variables or fields that cause performance problems can be determined precisely. 

\subsubsection{Flexible Reporting}
\label{sec:flexiblereport}

We provide two different ways to report those false sharing problems. Normally, we can report those false sharing problems in the end of a program. However, this way does not work for those long-running applications. Thus, we provide a on-demand way of reporting. User can send a specified signal to a corresponding applications. By intercepting those signals, we can report false sharing problems on demand. 

In order to report false sharing problems, we scan cache line status words of the globals and heaps and only report those false sharing problems that can possibly cause performance problems, based on a pre-defined threshold on the number of cache invalidations.  

\subsection{Detailed Implementations}

\label{sec:sheriffdetect}
We provide a tool, \SheriffDetect{}, to detect false sharing problems based on the \sheriff{} framework. The basic idea of \SheriffDetect{} is discussed in Section~\ref{sec:detectfalseshare}. More details is discussed in the following. 

\subsubsection{Tracking Memory Accesses}
\label{sec:memoryaccesses}

\begin{figure*}[!t]
\centering
\includegraphics[width=5in]{sheriff/figure/sheriffdetective.pdf}
\caption{
Overview of \SheriffDetect{}’s operation. \SheriffDetect{} extends \Sheriff{} with sampling, per-cacheline status arrays, and per-word status arrays. For clarity of
exposition, the diagram depicts just one cache line per page and two words per cache line.\label{fig:sheriffdetect}}
\end{figure*}

In order to track cache invalidations, we have to track memory accesses of different threads. When there is a memory access, we can check against its corresponding cache line status word to find out whether this memory access causes a cache invalidation or not. \SheriffDetect{} can only track memory writes so that it can only detect write-write false sharing problems. 

\sheriff{} framework provides a strong isolation of different threads' execution and only commits those changes of different threads to the shared mapping in the end of an transaction, by comparing a ``working'' page against its ``twin'' page as described in Section~\ref{sec:sherifftransaction}.
This implies that \sheriff{} is able to find those memory writes at synchronization points. 

However, if a transaction is long-running, finding those memory changes at the end of a transaction is not efficient to find those false sharing problems happening in the middle of a transaction. Actually, the \texttt{linear\_regression} benchmark (described in Section~\ref{sec:evaluation}), degrading the performance by more than $10\times$ because of its false sharing problem, only has a single transaction per thread. 

In order to detect memory accesses in the middle of an transaction, \SheriffDetect{} employs a sampling mechanism using the timer mechanism. When the timer is expired, \SheriffDetect{} tracks memory writes in the current period using the twinning and diffing mechanism. The detailed mechanism is described in Figure~\ref{fig:sheriffdetect}. 

To do this, \SheriffDetect{} also creates a ``temporary twin'' page for every page that have been accessed when the sampling timer is expired. Because these ``temporary twin'' pages are thread-private, we can create and update these pages in the timer handler by simply copying from their corresponding ``working'' pages. The difference between a ``working'' page and its ``temporary'' page implies at least one memory write in this sampling period. Currently, \SheriffDetect{} samples memory accesses of each thread at every 10 microsecond. The tradeoff between sampling period and performance is also discussed in Section~\ref{}. 

\subsubsection{Tracking Cache Invalidations}
\label{sec:invalidation}
As the discussion in Section~\ref{sec:detectionidea}, \SheriffDetect{} tracks and reports those cache lines with a big number of cache invalidations, where they are considered to cause serious performance problems. 

In order to track cache invalidations, \SheriffDetect{} introduces a cache line status word for every cache line of the globals and heap, showed in Figure~\ref{fig:cachelinestatusword}.  Every cache line status word contains two fields, the last thread writing to this cache line and the number of cache invalidations of this cache line. 
Every time, when \SheriffDetect{} detects a memory write, either at the end of transactions or in the sampling timer handler,  it updates these two fields correspondingly. Based on the assumptions described in Section~\ref{sec:detectionidea}, \SheriffDetect{} increments the number of cache invalidations when there is a write from a different thread. To avoid using lock, \SheriffDetect{} updates those counters using atomic primitives. 

\subsection{Optimizations}

\SheriffDetect{} also employs the following optimizations in order to reduce the performance overhead. 

\paragraph{Getting Callsite Information.}
\SheriffDetect{} intercepts memory allocation operations for collecting callsites for every heap object. To reduce the performance overhead, \SheriffDetect{} do not use the bracktrace(), but identify the callsite by analyzing return or frame address using GCC extensions. However, this can not work on applications without debugging information. 

\paragraph{Reducing timer overhead.}
As explained in Section~\ref{sec:memoryaccesses}, \SheriffDetect{} uses sampling to track cache invalidations. To reduce the impact of timer interrupts, \SheriffDetect{} activates sampling only when the average transaction time is larger than a threshold time (currently 10 milliseconds). \SheriffDetect{} uses an exponential moving average to track average transaction times ($\alpha = 0.9$). This optimization does not significantly reduce the possibility of finding false sharing since \SheriffDetect{}'s goal  is to find an object with a large amount of interleaved writes from different threads.

\paragraph{Sampling to find shared pages.} 
To reduce this overhead, \SheriffDetect{} leverages a
simple insight: if two threads can falsely share a cache line,
then they must simultaneously access the page containing
that cache line. \SheriffDetect{} relies on page protection to determine whether pages are shared or not. When one application has a large number of transactions or page touches, the protection overhead to gather this sharing information can dominate running time.

\SheriffDetect{} reduces overhead by using sampling to detect shared pages. If objects on a page are frequently falsely shared, the page itself must also be frequently shared, so even relatively infrequent sampling will eventually detect this sharing.  \SheriffDetect{} currently samples the first 50 out of every 1,000 periods (one period equals one transaction or one sampling interval). At the beginning of each sampled period, all memory pages are made read-only so that any
writes to each page will be detected. Upon finding a page is
shared, \SheriffDetect{} will track any false sharing inside it. \SheriffDetect{} only updates the shared status of pages during sampled periods and at commit points. During unsampled periods, pages whose sharing status is unknown impose no protection overhead.

\subsection{Limitation}
\label{discussion:faultofdetect}

Unlike previous tools, \SheriffDetect{} has no false positives, differentiates true sharing from false sharing, and avoid false positives caused by the reuse of heap objects.
\SheriffDetect{} can under-report false sharing instances in the following situations:

\paragraph{Single writer.}
False sharing usually involves updates from multiple threads, but it can also arise when there is exactly one thread writing to part of a cache line while other threads read from it. Because its detection algorithm depends on at least one differing update (that is, at least two writes of distinct values), \SheriffDetect{} cannot detect this kind of false sharing (though \sheriffprotect{} eliminates it; see Section~\ref{sec:patrol}).

\paragraph{Heap-induced false sharing.}  
\sheriff{} replaces the standard memory allocator with one that, like the Hoard allocator, avoids most heap-induced false sharing. \sheriff{}'s memory allocator (like Hoard), carves memory into page-sized chunks; each thread allocates
from its own set of chunks, and the allocator never splits cache lines across threads. Because \SheriffDetect{} uses the same allocator, it cannot detect false sharing that would be caused by the standard memory allocator. Since it is straightforward to deploy Hoard or a similar allocator to avoid heap-induced false sharing, this limitation is not a problem in practice.

\paragraph{Misses due to sampling.}  Since it uses sampling to
  capture continuous writes from different threads, \SheriffDetect{} can miss writes that occur in the middle of sampling intervals. We hypothesize that false sharing instances that affect performance are unlikely to perform frequent writes exclusively during that time, and so are unlikely to be missed.


\section{False Sharing Prediction}
% Why prediction is important?
\label{sec:prediction}
This section further motivates predictive false sharing and explains how to support it in the runtime system.  

\subsection{Overview}
%\begin{figure*}[!htb]
\label{sec:predictoverview}

\begin{figure}[!t]
\begin{center}
\includegraphics[width=6in]{predator/figure/perfsensitive}
\end{center}
\caption{
Performance of the linear\_regression benchmark (from Phoenix)  is highly sensitive to the memory layout between the (potentially) falsely-shared object and corresponding cache lines. 
\label{fig:perfsensitive}}
\end{figure}

The appearance of false sharing depends on 
the memory layout between objects and corresponding cache lines. The performance of a real example, linear\_regression, is shown in Figure~\ref{fig:perfsensitive}: 
When the offset of the starting address between the potentially falsely-shared object and corresponding cache lines is $0$ or $56$ bytes, there is no false sharing; 
When the offset is $24$ bytes, we see the most severe performance effect caused by false sharing. 
The performance difference caused by false sharing can affect the performance as large as $15\times$ on an 8-core machine. 

Existing detection tools can only report observed false sharing. That means, they may miss a very severe false sharing problem that could occur in the wild if the offset of the starting address was $0$ bytes or $56$ bytes in their test environment.
\Predator{} overcomes this shortcoming by accurately predicting potential false sharing, without the need of occurrences. 

\begin{figure*}
\begin{center} 
\subfigure[No false sharing]{%
   \label{fig:nofs}
   \includegraphics[width=0.24\textwidth]{predator/figure/Potential1}
}%
\hspace{10pt}
\subfigure[False sharing with larger cache size]{%
   \label{fig:fslarger}
   \includegraphics[width=0.24\textwidth]{predator/figure/Potential2}
}%
\hspace{10pt}
\subfigure[False sharing with different memory layout]{%
   \label{fig:fsnoalignment}
   \includegraphics[width=0.36\textwidth]{predator/figure/Potential3}
}%
\end{center}
%\includegraphics{fig/potential.pdf}
\caption{False sharing under different environments.}
\label{fig:potentialfalsesharing}
\end{figure*}

\Predator{} predicts {\it potential false sharing}, the type of false sharing that does not manifest in the current execution but may appear and greatly affect the performance of programs in a slightly different environment.

Figure~\ref{fig:potentialfalsesharing} shows a simplified example why occurrences of false sharing can change in different situations. 
In this figure, two rectangles with different patterns
represents two portions of the same object, updated by different threads. In Figure~\ref{fig:nofs}), there is no false sharing when thread T1 only updates 
``cache line 1'' and T2 only updates ``cache line 2''.
However, false sharing appears in one of the following cases, even with the same
access pattern. 

\begin{itemize}
\item
Doubling cache line size (Figure~\ref{fig:fslarger}). When the size of a cache line doubles, both T1 and T2 access the same cache line and false sharing occurs in this case.

\item
Different starting address of an object(Figure~\ref{fig:fsnoalignment}).  
When the starting address of this object is not aligned with the starting address of 
the first cache line, 
then T1 and T2 can update the second cache line simultaneously, 
causing a false sharing problem. 
%When some dynamic property changes the starting address of this object so that it 
%is not aligned with the starting address of the first cache line, 
\end{itemize} 

\Predator{} predicts whether programs can have potential false sharing in one of these two situations, where they can be caused by different dynamic properties. These dynamic properties include choosing different compiler, enabling different compiler optimizations, using different memory allocator, adding or removing code involving in memory allocations, changing different target platforms such as address mode (32-bit or 64-bit), and changing the size of cache line (64 Bytes or 128 Bytes). 
All dynamic properties, except changing the size of cache line, can lead to different memory layout, thus can possibly affect the occurrences of a false sharing problem. Thus, predicting false sharing in changing the memory layout or changing the size of cache line actually explores many possibilities caused by all of these dynamic properties.

\subsection{Basic Prediction Workflow}
\label{sec:predictionmechanism} 

%Similar to the detection part, 
\Predator{} focuses exclusively on potential false sharing that can 
cause performance problems.
The implementation is based on
two key observations. First, only accesses to 
adjacent cache lines can form potential false sharing, 
i.e., introducing cache invalidations when cache line size
or an object's starting address changes.
Second, only those cache lines with a large number of cache invalidations can degrade performance.

Based on these two observations, \Predator{} develops 
the following workflow to capture potential false sharing.
Those detection optimizations listed in Section~\ref{optimization} can also be applied
to the prediction part. We do not repeat these optimizations in this section.

\begin{enumerate}
\item
Track the number of writes on different cache lines. 

\item
When the number of writes to a cache line $L$ reaches {\it Tracking-Threshold},
\predator{} tracks the detailed read and write accesses for every word on both cache line $L$ 
and its adjacent cache lines. 

\item
When the number of writes to a cache line $L$ reaches a second threshold (called as
{\it Predicting-Threshold}), 
\predator{} identifies whether there exists false sharing in $L$ and its adjacent cache lines by analyzing word accesses information collected in Step 2, which are described in 
Section~\ref{sec:evaluatingfs} in detail.

\item
If a potential false sharing is found, \predator{} starts to track cache line invalidations in order to confirm its seriousness, which are discussed in Section~\ref{sec:tracking}.
Otherwise, go back to Step 2 to track more detailed accesses.
 
\end{enumerate}

\subsection{Searching for Potential False Sharing}
\label{sec:evaluatingfs}
To describe potential false sharing in two different cases, we first 
introduce the concept of a ``virtual cache line''.  A virtual cache line is a contiguous memory range that spans one or more physical cache lines.  In the case of double cache line size, a virtual line is composed of two originally contiguous cache lines, where it starts with a even number cache line.  Thus, only cache lines $2*i$ and $2*i+1$ can form a virtual cache line.  To evaluate a potential false sharing problem that can be caused by the change of memory layout, a virtual line should have the same size as physical lines, but with a  arbitrary starting addresss: unlike actual cache lines, the
starting address of a virtual cache line does not need to be multiple of the cache line size.  For instance, a 64-byte-long virtual line can consist of the range $[0,64)$ bytes or $[8,72)$ bytes.

To search for a potential false sharing problem, 
\Predator{} searches for a pair of hot accesses, one on $L$ and one on its previous or next cache line. Two accesses happening in the same actual cache line should be detected by the normal detection mechanism, thus we do not need to care about them. 

\predator{} analyzes the detailed information of word accesses collected in Step 2. A hot access refers to an access that has the number of read or write accesses larger than the average number of accesses. In fact, for every hot access $X$ in a specific cache line $L$, \Predator{} searches another
hot access $Y$ in $L$'s the previous cache line or next cache line, satisfying the following conditions: 

\begin{itemize}
\item
$X$ and $Y$ reside on the same virtual line. 

\item
One of $X$ and $Y$ is a write access.

\item 
$X$ and $Y$ are issued by different threads.

\end{itemize}

% why it finds a pair of $X$ and $Y$ == a potential false sharing 
Whenever it finds such a pair, $X$ and $Y$, 
\Predator{} identifies a potential performance-degrading false sharing problem since the number of cache invalidations possibly caused by $X$ and $Y$ (on a possible virtual line), 
can be larger than the average number of accesses. 
This approach is based on a similar observation as in detection: \emph{if a thread writes a virtual line after other threads have accessed the same virtual line, this write operation causes at least one cache invalidation}. 

However, before tracking detailed memory accesses on a virtual line, it is impossible to know exactly how many cache invalidations happen on a virtual line. Thus, \Predator{} conservatively assumes that accesses from different threads occurs in a interleaved way, with the maximum number of cache invalidations. This approach ensures that \Predator{} will not miss any potential false sharing as well as 
not reporting false positives.  

%According to above observation and assumption, 
%a pair of hot accesses, $X$ and $Y$, if accesses are issued in an interleaving 
%way, can generate the number of cache invalidations equaling to 
%the smaller number of accesses of $X$ and $Y$.
%Thus a false sharing problem is to be identified by \Predator{}.
  
After identifying possible false sharing, \Predator{} goes to Step 4 to verify whether this is an actual false sharing problem. 

\subsection{Verifying Potential False Sharing}
\label{sec:tracking}

\Predator{} verifies potential false sharing by tracking possible cache invalidations on a problematic virtual line.
%covering a pair of hot accesses found
%in Step 3.

For potential false sharing caused by double cache line size, as described in Section~\ref{sec:evaluatingfs}, a virtual line is always composed of cache line with index $2*i$ and $2*i+1$. 
\Predator{} tracks cache invalidations on the virtual line on which false sharing has been discovered.

However, for the case of a change in starting address,
two hot accesses with distance less than the cache line size 
can actually form multiple virtual lines. 
There is thus an additional step to determine which virtual line to be tracked.
Although the virtual line to be chosen here is never a real cache line of actual hardware because of unaligned addresses,
we utilize this virtual line to simulate the effect of changing the memory layout.


\begin{figure}
\begin{center} 
\includegraphics[width=6in]{predator/figure/trackpotential}
\end{center}
\caption{Determining a virtual line with size $sz$ according to hot accesses.}
\label{fig:trackpotential}
\end{figure}

Given two words with the hot accesses shown in Figure~\ref{fig:trackpotential}, \Predator{} leaves the same space before $X$ and after $Y$ in determining a virtual line to be tracked. That is, the virtual line starting at location $X-((sz-d)/2)$ and ending at $Y+((sz-d)/2)$ is tracked. 
This choice allows tracking of more possible cache invalidations caused by adjacent accesses of $X$ and $Y$. 
Since adjusting the starting address of a virtual line has the same effect of adjusting the starting address of an object in detecting false sharing, all cache lines related to the same object must be adjusted at the same time. \Predator{} then tracks cache invalidations based on these adjusted virtual lines.

In the end, \Predator{} can report accurately whether the change of memory layout can affect the performance or not, based on the possible number of cache invalidations. 

Currently, \predator{} only determines a specific virtual line to be tracked. However, we plan to choose different virtual lines in the future work, when the current choose cannot reveal a big number of cache invalidations. 



\section{Evaluation}
\label{sec:evaluation}


We perform our evaluation on an Intel Core 2 dual-processor CPU system, equipping with 16GB of RAM. Each processor is a 4-core 64-bit Xeon, running on at 2.33GHZ with a 4MB L2 cache.  The operating system is an unmodified CentOS 5.5, running with Linux kernel version 2.6.18-194.17.1.el5.

\subsection{Methodology}

We evaluate the performance and scalability of \dthreads{} versus CoreDet and \pthreads{} across the PARSEC~\cite{parsec} and Phoenix~\cite{phoenix-hpca} benchmark suites.  

In order to compare performance directly against CoreDet, which relies on the LLVM infrastructure~\cite{LLVM:CGO04}, all benchmarks are compiled with the LLVM compiler at the ``-O5'' optimization level~\cite{LLVM:CGO04}. Since \dthreads{} does not currently support 64-bit binaries, all benchmarks are compiled for 32 bit environments (using the ``-m32'' compiler flag). Each benchmark is executed ten times on a quiescent machine. To reduce the effect of outliers, the lowest and highest execution times for each benchmark are discarded,
so each result represents the average of the remaining eight runs.

\textbf{Tuning CoreDet:} 
The performance of CoreDet~\cite{Bergan:2010:CCR:1736020.1736029} is extremely sensitive to three parameters: the granularity for the ownership table (in bytes), the quantum size (in number of instructions retired), and the choice between full serial mode and reduced serial mode. We compare the performance and scalability of \dthreads{} with the best possible results that we could obtain for CoreDet on our system---that is, with the lowest average normalized
runtimes---after an extensive search of the parameter space (six possible granularities and 8 possible quanta, for each benchmark). The results presented here are for a 64-byte granularity and a quantum size of 100,000 instructions, in full serial mode.

\textbf{Unsupported Benchmarks}: We do not include results for 7 benchmarks from PARSEC, since they do not currently work with \dthreads{} (note that many of these also do not work for CoreDet). \texttt{vips} and \texttt{raytrace} would not build as 32-bit executables; \texttt{bodytrack}, \texttt{facesim}, and \texttt{x264} depend on sharing of stack variables;
\texttt{fluidanimate} uses ad-hoc synchronization, so it will not run without modifications; and \texttt{freqmine} does not use pthreads.

 
\textbf{Scalability Experiment}: For all scalability experiments, we logically disable CPUs using Linux's CPU hotplug mechanism, which allows us to disable or enable individual CPUs by writing ``0'' (or ``1'') to a special file (\texttt{/sys/devices/system/cpu/cpuN/online}).

\subsection{Determinism}

We first experimentally verify \dthreads{}' ability to ensure
determinism by executing the \emph{racey} determinism
tester~\cite{1508256}. This stress test contains, as its name
suggests, numerous data races and is thus extremely sensitive to memory-level non-determinism. \dthreads{} reports the same results for 2,000 runs. We also compared the schedules and outputs of all benchmarks used to ensure that every execution is identical.

\subsection{Performance}
\label{sec:performance}

\begin{figure*}[!t]
{\centering
\includegraphics[width=5in]{dthreads/figure/overhead-figure}
\caption{Normalized execution time with respect to \pthreads{} and CoreDet(lower is better). For 9 of the 14 benchmarks, \dthreads{} runs nearly as fast or faster than \pthreads{}, while providing deterministic behavior.\label{fig:performance}}
}
\end{figure*}

\begin{table*}[!t]
\centering
\begin{tabular}{l|rrr|rr|l}
{\bf \small Benchmark} & {\bf \small CoreDet} & {\bf \small \dthreads{}} & {\bf \small \pthreads{}} & $\frac{\mbox{\bf \small CoreDet}}{\mbox{\bf \small \pthreads{}}}$ & $\frac{\mbox{\small \bf \dthreads{}}}{\mbox{\small \bf \pthreads{}}}$ & {\bf \small Input} \\

\hline
{\bf \small histogram} & 0.97 & 0.73 & 0.35 & $1.32\times$ & $0.48\times$ & {\it \small large.bmp} \\
{\bf \small kmeans} & 68.41 & 13.16 & 15.02 & $5.20\times$ & $1.14\times$ & {\it \small -d 3 -c 1000 -p 100000 -s 1000} \\ 
{\bf \small linear\_regression} & 6.42 & 4.11 & 0.57  & $1.56\times$ & $0.14\times$ & {\it \small key\_file\_500MB.txt} \\
{\bf \small matrix\_multiply} & 31.68 & 19.32 & 19.28  & $1.63\times$ & $0.99\times$ & {\it \small 2000 2000 } \\
{\bf \small pca} & 39.24 & 20.49 & 21.14  & $1.92\times$ & $1.03\times$ & {\it \small -r 4000 -c 4000 -s 100 } \\
{\bf \small reverse\_index} & 7.85 & 2.06 & 6.53 & $3.81\times$ & $3.17\times$ & {\it \small datafiles} \\
{\bf \small string\_match} & 18.31 & 3.19 & 1.97 & $5.74\times$ & $0.62\times$ & {\it \small key\_file\_500MB.txt} \\
{\bf \small word\_count} & 17.17 & 2.17 & 2.37 & $7.91\times$ & $1.09\times$ & {\it \small word\_100MB.txt} \\
{\bf \small blackscholes} & 10.49 & 9.47 & 9.30 & $1.11\times$ & $0.98\times$ & {\it \small 8 in\_1M.txt prices.txt} \\
{\bf \small canneal} & 14.74 & 10.41 & 39.82 & $1.42\times$ & $3.83\times$ &  {\it \small 7 15000 2000 400000.nets 128} \\
{\bf \small dedup} & 3.38 & 1.45 & 5.39 & $2.33\times$ & $3.72\times$ & {\it \small -c -p -f -t 2 -i media.dat output.txt} \\
{\bf \small ferret} & 21.89 & 7.02 & 26.86 & $3.11\times$ & $3.83\times$ & {\it \small corel lsh queries 10 20 1 output.txt} \\
{\bf \small streamcluster} & 14.33 & 2.74 & 4.61 & $5.23\times$ & $1.68\times$ &  {\it \small 10 20 128 16384 16384 1000 none output.txt 8} \\
{\bf \small swaptions} & 35.21 & 4.18 & 3.88 & $8.42\times$ & $0.93\times$ & {\it \small -ns 128 -sm 50000 -nt 8} \\
\hline
\end{tabular}
\caption{Benchmarks: execution time (in seconds) and input parameters.\label{tbl:benchmarks}}
\end{table*}


We next compare the performance of \dthreads{} to CoreDet
and \pthreads{}. Figure~\ref{fig:performance} presents these results graphically (normalized to \pthreads{}); Table~\ref{tbl:benchmarks} provides detailed information about execution time and input parameters.

\dthreads{} outperforms CoreDet on 12 out of 14 benchmarks (running between 20\% and $11.2\times$ faster). For 9 benchmarks, \dthreads{} runs nearly the same as or higher
performance than \texttt{pthreads}. Because \dthreads{} isolates updates in separate processes, it can improve performance by eliminating false sharing: since concurrent ``threads'' actually execute in different physical pages, there is no coherence traffic caused by false sharing between synchronization points. \dthreads{} eliminates catastrophic
false sharing in the \texttt{linear\_regression} benchmark, allowing it to execute over $7\times$ faster than \pthreads{} and $11\times$ faster than CoreDet. The \texttt{string\_match} benchmark exhibits a similar, though less dramatic, false sharing problem, allowing \dthreads{} to run almost 60\% faster than \pthreads{} and $9\times$ faster than CoreDet. Two benchmarks, \texttt{histogram} and \texttt{swaptions}, also run faster with \dthreads{} than with \pthreads{} ($2\times$ and $6\%$, respectively; $2.7\times$ and $9\times$ faster than with CoreDet). We believe but have not yet verified that the reason is false sharing.

For some benchmarks, \dthreads{} incurs modest overhead. For example, unlike most benchmarks examined here, which create long-lived threads, the \texttt{kmeans} benchmark creates and destroys over 1,000 threads in the course of its execution. 
While Linux processes are relatively lightweight, creating and tearing down a process is still more expensive than the same operations of threads, accounting for a 14\% performance degradation of \dthreads{} over \pthreads{} (though it runs $4.6\times$ faster than CoreDet).

\dthreads{} runs substantially slower than \pthreads{} for 4 of the 14 benchmarks examined here. The \texttt{ferret} benchmark relies on an external library to analyze image files during the first stage in its pipelined execution model; this library makes intensive (and in the case of \dthreads{}, unnecessary) use of locks. Lock acquisition and release in \dthreads{} imposes higher overhead than ordinary \pthreads{} mutex operations. More importantly in this case, the intensive use of locks in one stage forces \dthreads{} to effectively serialize the other stages in the pipeline, which must repeatedly wait on these locks to enforce a deterministic lock
acquisition order. The other three benchmarks (\texttt{canneal}, \texttt{dedup}, and \texttt{reverse\_index}) modify a large number of pages. With \dthreads{}, each page modification triggers a segmentation violation, a system call to change memory protection, the creation of a private copy of the page, and a subsequent copy into the shared space on commit (see Section~\ref{sec:future-work} for planned optimizations that may reduce this cost). We note that CoreDet also substantially degrades performance for these benchmarks, so much of this slowdown may be inherent to any deterministic runtime system.

\subsection{Scalability}
\input{dthreads/scalability}


\subsection{Performance Analysis}

\subsubsection{Benchmark Characteristics}

The data presented in Table~\ref{tbl:characteristics} are obtained from the executions running on all 8 cores.  Column 2 shows the percentage of time spent in the serial phase.  In \dthreads{}, all memory commits and actual synchronization operations are performed in the serial phase.  The percentage of time spent in the serial phase thus can affect performance and scalability. Applications with higher overhead in \dthreads{} often spend a higher percentage of time in the
serial phase, primarily because they modify a large number of pages that are committed during that phase.

Column 3 shows the number of transactions in each application and Column 4 provides the average length of each transaction (ms).  Every synchronization operation, including locks, conditional variable, barriers, and thread exits, demarcate transaction boundaries in \dthreads{}.  For example, \texttt{reverse\_index}, \texttt{dedup}, \texttt{ferret}
and \texttt{streamcluster} perform numerous transactions whose
execution time is less than 1ms, imposing a performance penalty for these applications.  Benchmarks with longer (or fewer) transactions run almost the same speed as or faster than \texttt{pthreads}, including \texttt{histogram} or \texttt{pca}.  In \dthreads{}, longer transactions amortize the overhead of memory protection and copying.

Column 5 and 6 provides more detail on the costs associated with memory updates (the number and total volume of dirtied pages). From the table, it is clear why \texttt{canneal} (the most notable outlier) runs much slower with \dthreads{} than with \pthreads{}. This benchmark updates over 3 million pages, leading to the creation of private copies, protection faults, and commits to the shared memory space. Copying alone is quite expensive: we found that copying one gigabyte of memory takes approximately 0.8 seconds when using \texttt{memcpy}.

\textbf{Conclusion: }
Most benchmarks examined here contain either a small number or long running transactions, and modify a modest number of pages during execution. For these applications, \dthreads{} is able to amortize its overhead: by eliminating false sharing, it can even run faster than \pthreads{}. However, for the few benchmarks that perform numerous short-lived transactions, or modify a large amount of pages, \dthreads{} can introduce substantial overhead.


\begin{table*}[!t]
\centering
\begin{tabular}{l|rrrrr}
& {\bf \small Serial Phase} & {\bf \small Transactions} & {\bf \small TransLength} & {\bf \small DirtyPages} & {\bf \small DirtyPages}
\\
{\bf \small Benchmark} & {\bf \small (\% of time)} & {\bf (\#)} & {\bf \small (ms)} & {\bf \small (\#)} & {\bf \small (GB)}\\
%\hline
%\multicolumn{6}{|c|}{\emph{Phoenix}} \\
\hline
\small \textbf{histogram} & 0 & 23 & 15.47 & 29 & 0 \\
\small \textbf{kmeans} & 0 & 3929 & 3.82 & 9466 & 0.04\\
\small \textbf{linear\_regression} & 0 & 24 & 23.92 & 17 & 0\\
\small \textbf{matrix\_multiply} & 0 & 24 & 841.2 & 3945 & 0.02\\
\small \textbf{pca} & 0 & 48 & 443 & 11471 & 0.04 \\
\small \textbf{reverseindex} & 17\% & 61009 & 1.04 & 451876 & 1.72\\
\small \textbf{string\_match} & 0 & 24 & 82 & 41 & 0 \\
\small \textbf{word\_count} & 1\% & 90 & 26.5 & 5261 & 0.02\\
%\hline
%\multicolumn{6}{|c|}{\emph{PARSEC}} \\
%\hline
\small \textbf{blackscholes} & 0 & 24 & 386.9 & 991 & 0\\
\small \textbf{canneal} & 26.4\% & 1062 & 43 & 3606413 & 13.75\\
\small \textbf{dedup} & 31\% & 45689 & 0.1 & 356589 & 1.36\\
\small \textbf{ferret} & 12.3\% & 484127 & 0.05 & 844184 & 3.21 \\
\small \textbf{streamcluster} & 18.4\% & 130001 & 0.04 & 131992 & 0.50\\
\small \textbf{swaptions} & 0 & 24 & 163 & 867 & 0\\
\hline
\end{tabular}
\caption{Benchmark characteristics.\label{tbl:characteristics}}
\end{table*}

\subsubsection{Performance Impact Analysis}
We further evaluate the performance impact of two important components of \dthreads: deterministic synchronization (sync-only) and memory protection(prot-only).

\emph{Sync-only}: This configuration enforces a deterministic synchronization order. However, the memory protection is not enabled so different processes access the shared memory directly. We want to use this to show the performance impact of load imbalance, caused by synchronization based scheduling.

\emph{Prot-only}: This configuration runs threads in isolation, with commits at synchronization points. The order of synchronization and memory commits are non-deterministic. This configuration eliminates false sharing, but also introduces the performance overhead of isolation and memory commits. In order to guarantee correct execution, we replaced those synchronizations as corresponding cross-processes synchronizations, which are supported by \pthreads{} library. The lazy twin creation and single-threaded execution optimizations are disabled here because they are unsafe without deterministic synchronization.


\begin{figure*}[!t]
{\centering
\includegraphics[width=6.25in]{dthreads/figure/perfeffect}
\caption{Normalized execution time with respect to \pthreads{} (lower is better) for three different configurations. 
\label{fig:perfanalysis}}
}
\end{figure*}

The performance results of these two configurations are shown in Figure~\ref{fig:perfanalysis} and discussed in the following.

\begin{itemize}

\item
The \texttt{reverse\_index}, \texttt{dedup} and \texttt{ferret} benchmarks show significant load imbalance with {\it sync-only} configuration. Additionally, these benchmarks introduces significant overhead with {\it prot-only} configuration because of a large number of transactions there. That explains why \dthreads{} doesn't have good performance on these benchmarks.

\item
The \texttt{string\_match} and \texttt{histogram} benchmarks show performance improvement with {\it sync-only} configuration. The exact reason is not clear, may be due to the per-thread allocator. 

\item
The \texttt{linear\_regression}, \texttt{histogram} and \texttt{swaptions} benchmarks improve performance with {\it prot-only} configuration. The memory isolation mechanism eliminates the false sharing problem inside and contributes to the performance speedup.

\item
Normally the performance of \dthreads{} is not better than the performance of {\it prot-only} configuration. However, both \texttt{ferret} and \texttt{canneal} run faster with determinism enabled. Both are benefited from specific optimization described in Section~\ref{sec:dthreads-optimization}. \texttt{ferret} benefits from the \emph{single-threaded-execution}. The performance improvement of \texttt{canneal} is coming from shared twin pages for all threads in the phase.

\end{itemize}








