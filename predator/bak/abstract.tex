%How is existing work?
%Sharing inside mulithreaading programs is not easy, they can easily cause correctness or performance problem. 
%Inappropriate sharing can dramatically degrade the performance of 
%mulithreading programs and seriously affect the scalability. 
%So detecting false sharing accurately and precisely can be helpful for user to fix corresponding performance problem. 


False sharing is a notorious problem for multithreaded applications
that can drastically degrade both performance and
scalability. Existing false sharing detectors can precisely identify
the sources of false sharing, but are limited to reporting false
sharing actually observed during execution: they do not generalize
across executions. Because false sharing is extremely sensitive to
object layout, these detectors can easily miss false sharing problems
that can arise due to slight differences in memory allocation order or
object placement decisions by the compiler. In addition, they cannot
predict the impact of false sharing on hardware with different cache
line sizes.

%objects and cache lines: any change of compiler optimization, compiler, memory manager, 
%memory allocation order, cache line size or different target binary 
%may change alignments, and thus affect occurrences of false sharing, 
%which leaves many of them undetected by existing tools.

This paper presents \Predator{}, a predictive software-based false
sharing detector. \Predator{} generalizes from a single execution to
precisely predict false sharing that is latent in the current
execution. \predator{} tracks accesses within a range that could lead
to false sharing given different object placement. It also tracks
accesses within
\emph{virtual cache lines}, contiguous memory ranges that span actual
hardware cache lines, to predict sharing on hardware platforms with
larger cache line sizes. For each, it reports the exact program
location of predicted false sharing problems, ranked by their
projected impact on performance. We evaluate \Predator{} across a
range of benchmarks and actual applications: \Predator{} identifies
problems undetectable with previous tools, including two
previously-unknown false sharing problems, with no false
positives. \Predator{} located false sharing problems in MySQL and the
Boost library that had eluded detection for years.

%\Predator{} identified two unknown false sharing problems 
%Besides, \Predator{} have successfully detected false sharing of real applications,
%including \texttt{mysql} server application and \texttt{boost} library. Fixing these
%false sharing problems improves performance by $6\times$ and $40\%$ correspondingly.



%False sharing is a notorious performance issue for different software stacks, 
%which can dramatically degrade the performance and seriously affect the scalability of 
%systems.

%Many reserach efforts have been made to detect false sharing. 
%Unfortunately, previous approaches to detect false sharing
%either introduce significant performance overhead, or fail
%to report false sharing accurately and precisely, or have different limitations of usage. 
%\sheriff{}, the prior state-of-the-art tool, 
%can only detect write-write false sharing in applications using \pthreads{} library.
%This paper presents a novel approach, \Predator{}, to combine compiler instrumentation
%and runtime system to detect false sharing. 
%the compiler instruments every memory access and 
%the runtime system collects and analyzes memory accesses to detect false sharing problems.
%Since it does not rely on any hardware, OS or threading library, this approach can be
%applied to the entire software stack without any limitation. 
%\Predator{} can detect false sharing accurately and precisely: it reports no 
%false positives and pinpoints exact objects with false sharing problems.
%Also, unlike previous work, this method can be extended to
%identify false sharing problems across the entire software stack, including 
%hypervisors, operating systems, libraries and applications. 
%Experimental results on two popular benchmark suites 
%show that \Predator{} not only detected all known false sharing problems but also revealed 
%two unknown false sharing problems.
%Besides, \Predator{} have successfully detected false sharing of real applications,
%including \texttt{mysql} server application and \texttt{boost} library. Fixing these
%false sharing problems improves performance by $6\times$ and $40\%$ correspondingly.

%Moreover, existing tools can only detect those manifested false sharing problems.
%However, occurrences of false sharing can be affected by alignments between
%objects and cache lines: any change of compiler optimization, compiler, memory manager, 
%memory allocation order, cache line size or different target binary 
%may change alignments, and thus affect occurrences of false sharing, 
%which leaves many of them undetected by existing tools.
%\Predator{} is the first tool which can accurately predict possible false sharing 
%without the need of occurrences. 
%It can report all false sharing problems with only one execution and with reasonable overhead, 
%around $6.7\times$ performance overhead on average.

%What is novel in our work?
%How is the performance overhead?
