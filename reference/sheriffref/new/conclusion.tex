\label{sec:conclusion}
This paper presents two tools that attack the problem of false sharing
in multithreaded programs. Both are built with \sheriff{}, a
software-only framework that enables per-thread memory protection and
isolation. \sheriff{} works by converting threads into processes, and
uses memory mapping and a difference-based commit protocol to provide
isolated writes. \sheriffdetect{} identifies false sharing instances
with no false positives, and pinpoints the objects involved in
performance-critical false sharing problems. We show
that \sheriffdetect{} can greatly assist programmers in tracking down
and resolving false sharing problems. When it is not feasible for
programmers to resolve these problems, either because code is
unavailable or because the fixes would degrade performance
further, \sheriffprotect{} can be used to automatically eliminate the
false sharing problems indicated by \sheriffdetect{}. We show
that \sheriffprotect{} can substantially improve the performance of
applications with moderate to severe false sharing, without the need
for programmer intervention or source code.

We plan to release the source code for \sheriff{}, \sheriffdetect{},
and \sheriffprotect{} by publication time.
