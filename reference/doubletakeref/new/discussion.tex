In this section, we discuss \doubletake{}'s limitations, the limitations of the dynamic analyses we have implemented, and our plans for future work on \doubletake{}.

\subsection{Limitations}

\doubletake{} lets dynamic analyses run with very low overhead, then
re-execute with heavyweight instrumentation when an error has been
detected. This approach will not work for errors that are not
monotonic: once an invariant has been violated, its evidence must
remain until the end of the epoch.

\subsubsection*{Applications}
We have used \doubletake{} to implement detectors for heap buffer
overflows, use-after-free errors, and memory leaks. These tools
demonstrate \doubletake{}'s effectiveness as a framework for efficient
dynamic analyses, but they are not perfect. Our heap buffer overflow
detector cannot identify all non-contiguous buffer overflows. If an
overflow touches memory only in adjacent objects and not
canaries, \doubletake{}'s end-of-epoch scan will not find any evidence
of the overflow. Both the buffer overflow and use-after-free detectors
can detect errors only on writes. To reduce overhead, use-after-free
detection only places canaries in the first 128 bytes of freed
objects. If a write to freed memory touches only above this threshold,
our detector will not find it. The memory leak detector will not
produce false positives, but non-pointer values that look like
pointers to leaked objects can lead to false negatives. Finally, if a
leaked object was not allocated in the current epoch, \doubletake{}'s
re-execution will not be able to find the object's allocation site. In
practice, \doubletake{}'s epochs are long enough to collect allocation
site information for all leaks detected during our evaluation.
 
\subsection{Future Work}
We plan to explore further analyses directions, including concurrency error detection.

To further reduce \doubletake{}'s overhead, we plan to
replace \doubletake{}'s custom allocator with a more efficient
heap. For use-after-free detection, large objects could be allocated
directly using \texttt{mmap} and protected rather than filling them
with canaries.

We also plan to integrate the \doubletake{} framework
with \texttt{gdb}. Lightweight rollback and re-execution would be
useful for diagnosing application errors during a debugging
session. Additionally, a program run with \doubletake{} could
automatically begin a \texttt{gdb} session when an error is detected.
