% Dynamic analysis
% Super awesome

Dynamic analysis tools are widely used to find bugs in
applications. They are popular among programmers because of their
precision---for many analyses, they report no false positives---and
can pinpoint the exact location of errors, down to the individual line
of code.

Perhaps the most prominent and widely used dynamic analysis tool for
C/C++ binaries is Valgrind~\cite{}. Valgrind's most popular use case
is to check memory errors, including buffer overflows, dangling
pointer errors, and memory leaks.

% Used for lots of tasks.



% BUT: expensive. Can't really use it all the time.
% Makes debugging a bitch.
% So now what?




Buffer overflows are still one of dominant errors 
even after decades of research in this field~\cite{}: 
Four types of the CWE/SANS "Top 25 Most Dangerous Software Errors" 
~\cite{overflows1} and 
over one-third valnuerabilites of 1818 high serverity problems
reported from NIST~\cite{overflows2} are related to buffer overflows.

% What is the state-of-the-art? Why it is necessary to have our tool?
Since it is impossible to for programmers to write error-free programs, 
existing these problems are mainly caused by imperfection
of detection tools, which preventing the usage of them: 
either existing tools are not easy-to-use or they can not provide precise information 
or their performance
overhead are too high to be used.
A lot of tools needs programs to be annotated~\cite{} or to be recompiled~\cite{}, 
which is inconvenient for legacy software or software using third-party libraries.
Besides that, they can not provide enough information to fix problems: they can
at best point out those statements with buffer overflow problems, but without the context
why those statements are executed~\cite{}. 
Also, the performance overhead of existing tools are still too high to be
used in the production environment: two state-of-the-art tools~\cite{} are imposing over 25\%
performance overhead. 

\subsection{Contribution}
\doubletake{} is designed to tackle all these problems simultaneously for \textbf{single-threaded}
programs and it has the following contributions:

\begin{itemize}
\item \textbf{Easy-to-use}:
\doubletake{} is a drop-in library: there is no need to change the
existing operating system, to change the source code or to recompile a program,
we can simply link a program to \doubletake{} library directly or 
use the LD\_PRELOAD mechanism to link \doubletake{} beforehand.

\item \textbf{Efficiency}:
\doubletake{} designes to reduce the performance overhead for 
programs with no buffer overflows (the common case). For those programs with buffer overflows, 
\doubletake{} rollbacks and re-execute a program to locate the problems. 
With averagely 8\% performance overhead, the performance of \doubletake{} is much better than 
that of state-of-the-art tools, where imposing at least 25\% performance overhead.

\begin{comment}
Existing tools normally stop the program when an overflow is detected so that they can report 
where a problem occurs. But to detect a overflow immediately,
they instrument every memory access and introduce significant 
performance overhead (by more than 23\%). 
Completely different with these tools, \doubletake{} designes to reduce the performance overhead for 
programs with no buffer overflows (the common case) and re-executes a program deterministically
after installing watch point for those programs with buffer overflows. 
With averagely 2\% performance overhead, the performance of \doubletake{} is even better than   
that of Cruiser~\cite{overflow:Cruiser} (5\% slowdown): Cruiser utilizes an additional CPU 
to run an overflow checking thread simulataneously but can not report the locations of problems.
\end{comment} 

\item \textbf{Precision}:
\doubletake{} can report very precise and detailed information about buffer overflow problems:  
\doubletake{} reports the context of problematic memory addresses, 
which existing tools at best point out those statements with buffer overflow problems.

\item \textbf{Reliability}:
\doubletake{} does not report any false positive: all buffer overflow problems reported 
are actual buffer overflow problems. 
Besides, \doubletake{} can still work correctly even facing with segmentation fault errors.
Most of tools if not all crash unexpectedly and silently, 
without providing any useful information in case of segmentation faults 
caused by some buffer overflow and dangling pointers.
%\doubletake{} can survive these memory errors by intercepting the segmentation fault: \doubletake{}
%can even report the reason causing the segmenation fault.
\doubletake{} can find out all possible buffer overflows at a time. 
There is no need to find buffer overflows one by one: current tools
always stopped at the first error.

\item \textbf{Completeness}:
Approaches checking every memory access (relying on compiler or binary instrumentation)
may miss some buffer overflows caused by uninstrumented libraries and system calls. 
\doubletake{} can detect all kinds of buffer overflows, including those overflows caused by
system call and libraries.
  
\end{itemize} 


In order to detect a overflow immediately and report to users where a problem occurs,
existing tools instrument every memory access and introduce significant 
performance overhead (by more than 23\%). 
Completely different with these tools, \doubletake{} designes to reduce the performance overhead for 
programs with no buffer overflows (the common case) and re-executes a program when necessary to locate 
programs. 
%whenever buffer overflows or other memory errors are captured, 
%\doubletake{} rollbacks this program to its previous state 
%and re-executes it after installing hardware watchpoints 
%on problematic memory addresses~\cite{}.

In the beginning, \doubletake{} snapshots the state of a program by saving program counter,
registers, stack, heap state and globals. 
When buffer overflows or other memory errors are detected, \doubletake{} rollbacks a program
to its previous state: it recovers the stack, heap and globals from the saved state,
then it sets the context to saved registers (including program counter),
 making the program to be re-executed.
In order to detect those memory accesses causing buffer overflows or other errors, 
\doubletake{} installs watch points on those problematic memory addresses. 
  
For programs with no overflows(the common case), \doubletake{} only introduces the overhead 
of snapshotting a program in the beginning and checking overflows on all heap objects 
in the end, program running in the full speed most of time,  
thus it keeps the overhead to a minimum.
By using the watchpoint mechanism, \doubletake{} can report the context of buffer overflows, by presenting the whole accesses on problematic addresses, helping the user to figure out reasons for memory errors.

\subsection{Outline}
This paper talks about those related works in next section. Section 3 discusses specific mechanisms 
used by \doubletake{}. Section 4 describes some implementation details worth noting. 
Section 5 evaluates the performance and effectiveness of \doubletake{} and Section 6
discusses some problems and extending possibilities of \doubletake{}.
Section 7 concludes this paper. 
