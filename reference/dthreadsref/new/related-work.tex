\label{sec:related-work}

The area of deterministic multithreading has seen considerable recent
activity. Due to space limitations, we focus here on software-only,
non language-based approaches.

% ~\cite{Bocchino:2009:TES:1640089.1640097,Burckhardt:2010:CPR:1869459.1869515,Simpson:1999:SEE:330346.330357}

Grace prevents a wide range of concurrency errors, including
deadlocks, race conditions, ordering and atomicity violations by imposing
sequential semantics on threads with speculative execution~\cite{grace}.  \dthreads{} borrows Grace's
threads-as-processes paradigm to provide memory isolation, but differs from Grace in terms of semantics, generality, and performance.

Because it provides the effect of a serial execution of all threads,
one by one, Grace rules out all interthread communication, including
updates to shared memory, condition variables, and barriers. Grace
supports only a restricted class of multithreaded programs: fork-join
programs (limited to thread create and join). Unlike
Grace, \dthreads{} can run most general-purpose multithreaded programs
while guaranteeing deterministic execution.

\dthreads{} enables far higher performance than Grace for several reasons:
It deterministically resolves conflicts, while Grace must rollback and re-execute threads that update any shared pages (requiring code modifications to avoid serialization); 
\dthreads{} prevents false sharing while Grace exacerbates it; and 
\dthreads{} imposes no overhead on reads.

CoreDet is a compiler and runtime system that represents the current
state-of-the-art in deterministic, general-purpose software
multithreading~\cite{Bergan:2010:CCR:1736020.1736029}. It uses
alternating parallel and serial phases, and a token-based global
ordering that we adapt for \dthreads{}. Like \dthreads{}, CoreDet
guarantees deterministic execution in the presence of races, but with
different mechanisms that impose a far higher cost: on average
$3.5\times$ slower and as much as $11.2\times$ slower than \dthreads{} (see
Section~\ref{sec:evaluation}). The CoreDet compiler instruments all
reads and writes to memory that it cannot prove by static analysis to
be thread-local.  CoreDet also serializes \emph{all} external library
calls, except for specific variants provided by the CoreDet runtime.

% \dthreads{} does not serialize library calls unless they perform 
% synchronization operations, and only traps on the first write to a page during
% a transaction.  Because of these differences, CoreDet runs 

CoreDet and \dthreads{} also differ semantically. \dthreads{}
only allows interleavings at synchronization points, but CoreDet relies on
the count of instructions retired to form quanta. This approach makes
it impossible to understand a program's behavior by examining the
source code---the only way to know what a program does in CoreDet (or
dOS and Kendo, which rely on the same mechanism) is to execute it on
the target machine. This instruction-based commit schedule is also
brittle: even small changes to the input or program can cause a
program to behave differently, effectively ruling out {\tt printf}
debugging. \dthreads{} uses synchronization operations as
boundaries for transactions, so changing the code or input does not
affect the schedule as long as the sequence of synchronization
operations remains unchanged.  We call this more stable form of determinism {\em robust determinism}.

% The use of synchronization points as commit boundaries also makes \dthreads{}
% code relatively \emph{robust}: when updates occur after a given number of 
% instructions retired (as in CoreDet and Kendo), it is impossible for 
% programmers to know when interleavings can occur. Such boundaries could vary 
% depending on the underlying architecture and would also be input-dependent, 
% meaning that slightly different inputs could lead to dramatically different
% thread interleavings. By contrast, \dthreads{} guarantees that only changes to
% the sequence of synchronization operations affect the order in which updates 
% are applied.

dOS~\cite{deterministic-process-groups} is an extension to CoreDet
that uses the same deterministic scheduling framework.  dOS provides
deterministic process groups (DPGs), which eliminate all internal
non-determinism and control external non-determinism by recording and
replaying interactions across DPG boundaries. dOS is orthogonal and
complementary to \dthreads{}, and in principle, the two could be
combined.

Determinator is a microkernel-based operating system that enforces
system-wide determinism~\cite{efficient-system-enforced}.  Processes
on Determinator run in isolation, and are able to communicate only at
explicit synchronization points.  For programs that use condition variables,
Determinator emulates a legacy thread API with quantum-based determinism similar
to CoreDet.  This legacy support suffers from the same performance and robustness problems as CoreDet.

Like Determinator, \dthreads{} isolates threads by running them in separate processes, but natively supports all \pthreads{} communication primitives.  \dthreads{} is a drop-in replacement for \pthreads{} that needs no special operating system support.

Finally, some recent proposals provide limited determinism. Kendo
guarantees a deterministic order of lock acquisitions on commodity
hardware (``weak determinism''); Kendo only enforces full (``strong'')
determinism for race-free
programs~\cite{1508256}. TERN~\cite{stable-deterministic} uses code
instrumentation to memoize safe thread schedules for applications, and
uses these memoized schedules for future runs on the same
input. Unlike these systems, \dthreads{} guarantees full determinism even
in the presence of races.

% \subsection{Other Related Work}

% Behavior-oriented parallelism (BOP)~\cite{1250760}.

% Transactional memory?

% Deterministic Record/Replay system: it is a different part, most of replay
% system's target is for debugging. In Deterministic programming language:
% Race detection.

%Languages.

