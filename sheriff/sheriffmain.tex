\label{sec:intro}

Multithreading is considered as one way to take advantage of those 
computation resources provided by multicore processors by spawning multiple threads 
within the context of a single process~\cite{multithread}. 
Unfortunately, writing efficient multithreaded programs are still a challenging task. 
False sharing are one of major source of performance problem. 
False sharing occurs when different threads (on different cores) are accessing on
different words in the same cache line. 
Cache coherence protocol forces one cache line to invalidate if some part of this cache line
has been modified by another processor. False sharing can force one core to wait for 
unnecessary updates from another processor, thus waste the CPU time and precious memory bandwidth. 
False sharing is a well-known performance issue on multi-core machine with 
separate caches~\cite{falseshare:Analysis, falseshare:effect}. 
One microbenchmark shows that false sharing can degrade the performance 
down to 160X slower (see Fig.~\ref{fig:benchmark2}).

In order to detect the false sharing problems,
there are some approaches
by using simulator~\cite{falseshare:simulator}, 
binary instrumentation technique~\cite{falseshare:binaryinstrumentation1, falseshare:binaryinstrumentation2}, 
or relying on performance monitor unit hardware~\cite{detect:ptu, detect:intel}.
But those approaches can suffer from numerous false postives 
or significant performance overhead (about 200X slower). 

Besides that, all previous approaches has the same problems. 
First, they have false positives (see Fig.~\ref{fig:benchmark5} for an example) 
on dynamic heap objects(one major source of false sharing problems).
Since they don't intercept memory allocation or de-allocation operations, those tools will un-correctly 
aggregate objects access information when one address is re-used by multiple objects.
Second, all of previous tools fail to pinpoint the source of false sharing problems, 
at best pointing to particular addresses accessed by functions,
which still need a lot of manual effort to find where and why false sharing occurs.

Sheriff is designed to overcome all previous shortcomings metioned above. 
Sheriff is a software-only system (relying on hardware's page fault mechanism only) 
which doesn't rely on advanced hardware support to find the problem.
Sheriff is based on a runtime system which replaces the pthread library and can eleminating 
false sharing silently.
Sheriff won't introduce any false positives, thus programmer don't need any
unnecessary work to confirm that problem. 
Sheriff can precisely locate false sharing problem, 
by providing name and size information for global variables and providing allocation sites for heap objects, 
thus it is easy to fix the false sharing problem by these information (see one case study in 
Section~\ref{evaluation:comparison}). 
Besides, the performance overhead of Sheriff is only 26\% (???) on average, which is notably lower than
previous tools using binary instrumentation 
technique~\cite{falseshare:binaryinstrumentation1, falseshare:binaryinstrumentation2}.

\section{Framework}
\label{sec:overview}

\begin{figure}[!t]
\begin{center}
\includegraphics[width=3.3in]{doubletake/figure/overview}
\end{center}
\caption{
Overview of \doubletake{}: execution is divided into epochs at the boundary of irrevocable system calls. 
\label{fig:overview}}
\end{figure}

\doubletake{} aims to reduce the performance overhead of dynamic analyses for memory errors sharing the monotonicity: the evidence of an error is persistent and can be detected after-the-fact. As described in Figure~\ref{fig:overview}, the execution of a program is divided into epochs at irrevocable system calls, discussed in Section~\ref{sec:normal_execution}. Inside each epoch, \doubletake{} allows a program to run at full speed, with the support of checkpointing and very minimum recording overhead. \doubletake{} only checks for evidence of an error when an epoch ends. If an error is detected, \doubletake{} rolls back to the most recent checkpoint and re-execute the program to pinpoint the error.  During the re-execution phase, \doubletake{} can use higher-overhead mechanisms to pinpoint the exact cause of the error. 

Based on this framework, we have implemented three detection tools for heap buffer overflows, use-after-free errors, and memory leaks, which are described in detail in Section~\ref{sec:applications}. 

\doubletake{} employs the following core mechanisms:

\paragraph{Efficient Recording.}
At the beginning of every epoch, \doubletake{} saves a snapshot of program registers, and all writable memory. The epoch ends when the program attempts to issue an irrevocable system call, which is described in Section~\ref{} \CC{here}. \doubletake{} also records the order of thread synchronization operations to support re-execution of parallel programs. \doubletake{} records minimal system state at the beginning of each epoch (like file offsets), which allows system calls that modify this state to be undone if re-execution is required. As a result, most programs require very few epochs and program state checks. We describe the details of each application's state checks in Section~\ref{sec:applications}.

\paragraph{Precise Replay.}
During re-execution, \doubletake{} ensures that all observable system state, system call results, and memory allocations will be the same from the original run. System calls that cannot be replayed, called irrevocable system calls, must be issued at the end of an epoch after detection tools have verified that no errors have occurred. Actually, those system calls consists of the boundary of epochs, which are discussed in Section~\ref{sec:normal_execution}. In practice, most system calls can easily be replayed by handling appropriately. This ensure that most programs run very few integrity checks, and overhead remains low when errors are not detected.

\paragraph{Custom Heap Allocator.}

\doubletake{} replaces the default heap allocator with a BiBOP-style memory allocator, built on HeapLayers~\cite{heaplayers}. \DoubleTake{} pre-allocates a fixed size of memory from its underlying operating system using \texttt{mmap} system calls and satisfies memory allocations from this block by interposing all memory allocation and deallocations. Using the custom memory allocators avoids a big number of sbrk() or mmap() system calls caused by the default memory allocator. In the heap, all heap objects have the block size of {\it power of $2$}, using an object header to mark its status and size information. There is no split and merge operation on heap objects. If the size of an allocation is less than {\it power of 2}, \DoubleTake{} allocates an object with the size of next {\it power of 2}. 






\section{Detecting False Sharing}
\label{sec:detectfalseshare}

This section first describes the basic idea of detecting false sharing. 

\subsection{Basic Idea}
\label{sec:detectionidea}

There is a false sharing problem when two threads simultaneously access independent data in the same cache line. False sharing does not necessarily cause performance problems. It can greatly degrade performance only when those accesses, caused by threads running on different cores with separate cache, actually cause a big number of cache invalidations. This is our \textbf{basic observation}. 

However, it is hard to know where a thread is located in the user space. Moreover, it is unnecessary to know this relationship since the relationship between threads and cores can be changed from one execution to the other. Thus, we make the following \textbf{assumptions}. First, we conservatively assume that all threads are running on different cores, with separate caches. Thus we can report the worst-case results caused by false sharing problems. In order to avoid expensive cache simulation, which can be affected by cache hierarchy, cache capacity and cache eviction rule, we further assume that the data is never evicted from its private cache by cache eviction, with infinite cache. This second assumption allows us to predict the cache invalidations based on memory accesses only, no need to think about cache eviction.   

Based on our basic observation, detecting false sharing problems turns into locating cache lines with a big number of cache invalidations. Cache invalidations are invoked by cache coherence protocol, the protocol used to ensure the coherence of data in the multiple-processor system. Based on our assumptions, there is a cache invalidation if a processor writes a cache line after another thread's access on the same cache line. Because the last thread accessing this cache line creates a copy of the same cache line on its running core's private cache (first assumption) and  holds this copy(second assumption), this write operation definitely cause a cache invalidation before or after this write operation, depending on different cache coherence protocols. 

\begin{figure}[!t]
\centering
\includegraphics[width=5in]{fig/cachelinestatuswords}
\caption{
To detect false sharing, each cache line of the globals and heap maintains a cache line status word, which is updated on each tracked memory access. \label{fig:cachelinestatusword}}
\end{figure}


To locate cache lines with a big number of cache invalidation, we maintain a cache line status word for each cache line in the globals and heap, shown in Figure~\ref{fig:cachelinestatusword}. We share the similar mechanism as another concurrent work of Zhao et.al. ~\cite{qinzhao}. However, the detailed implementation is totally different. Zhao et.al. utilize the detailed ownership bitmap and last thread bitmap to precisely track the cache invalidation, which can even track how many cache invalidations may happen in a write operation. However, their design can not easily scale to more than 32 threads, requiring more memory overhead caused by more bits and more checking performance overhead. Also, their approach miss one important factor - how many cache invalidations happening on a specific cache line. Without this information, it is impossible to pinpoint false sharing problems that can cause performance problems.   
Our approach overcomes all these shortcomings, by only tracking the last thread index and the number of cache invalidations. Thus, we can rank the seriousness of false sharing problems based on the number of cache invalidations. 

\subsubsection{Accurate Detection}
\label{sec:accuratedetect}

Accurate detection implies that we only report those false sharing problems that can cause performance problems. 

First, we only report those problems that can cause performance problems. This has been resolved by only reporting false sharing with a big number of cache invalidations. 

Second, this also implies that we should be able to differentiate false sharing from true sharing. As we all know, true sharing also can cause cache invalidations, which is the essence to have coherence protocol. In order to differentiate false sharing with true sharing, we further  track word-level access information for those cache lines involved in false sharing: how many reads or writes to each word by which thread. When a word is accessed by multiple threads, we mark the origin of this word as a shared access and do not track threads for further accesses to it. This information lets us accurately distinguish false sharing from true sharing in the reporting phase. It also helps diagnose where actual false sharing occurs when there are multiple fields or multiple objects in the same cache line, as this can greatly reduce the manual effort required to fix the false sharing problems.
  
Third, we should avoid pseudo false sharing (false positives) caused by memory reuses.  We update recording information at memory de-allocations for those objects without false sharing problems; heap objects involved in false sharing are never reused so that they can be reported in the end or on demand. 

\subsubsection{Precise Detection}
\label{sec:precisedetect}

Precise detection implies that we can precisely point out where the problem is. Thus, programmers can leverage on that to identify and correct false sharing. 

We identify globals directly by using debug information that associates the address with the name of the global. In order to precisely report the origins of heap objects with false sharing problems, we collect callsite information for each heap object by intercepting memory allocations and deallocations, and report to users about the origins of false sharing objects. 

Also, we present the word-level accesses information to users so that the exact variables or fields that cause performance problems can be determined precisely. 

\subsubsection{Flexible Reporting}
\label{sec:flexiblereport}

We provide two different ways to report those false sharing problems. Normally, we can report those false sharing problems in the end of a program. However, this way does not work for those long-running applications. Thus, we provide a on-demand way of reporting. User can send a specified signal to a corresponding applications. By intercepting those signals, we can report false sharing problems on demand. 

In order to report false sharing problems, we scan cache line status words of the globals and heaps and only report those false sharing problems that can possibly cause performance problems, based on a pre-defined threshold on the number of cache invalidations.  

\subsection{Detailed Implementations}

\label{sec:sheriffdetect}
We provide a tool, \SheriffDetect{}, to detect false sharing problems based on the \sheriff{} framework. The basic idea of \SheriffDetect{} is discussed in Section~\ref{sec:detectfalseshare}. More details is discussed in the following. 

\subsubsection{Tracking Memory Accesses}
\label{sec:memoryaccesses}

\begin{figure*}[!t]
\centering
\includegraphics[width=5in]{sheriff/figure/sheriffdetective.pdf}
\caption{
Overview of \SheriffDetect{}’s operation. \SheriffDetect{} extends \Sheriff{} with sampling, per-cacheline status arrays, and per-word status arrays. For clarity of
exposition, the diagram depicts just one cache line per page and two words per cache line.\label{fig:sheriffdetect}}
\end{figure*}

In order to track cache invalidations, we have to track memory accesses of different threads. When there is a memory access, we can check against its corresponding cache line status word to find out whether this memory access causes a cache invalidation or not. \SheriffDetect{} can only track memory writes so that it can only detect write-write false sharing problems. 

\sheriff{} framework provides a strong isolation of different threads' execution and only commits those changes of different threads to the shared mapping in the end of an transaction, by comparing a ``working'' page against its ``twin'' page as described in Section~\ref{sec:sherifftransaction}.
This implies that \sheriff{} is able to find those memory writes at synchronization points. 

However, if a transaction is long-running, finding those memory changes at the end of a transaction is not efficient to find those false sharing problems happening in the middle of a transaction. Actually, the \texttt{linear\_regression} benchmark (described in Section~\ref{sec:evaluation}), degrading the performance by more than $10\times$ because of its false sharing problem, only has a single transaction per thread. 

In order to detect memory accesses in the middle of an transaction, \SheriffDetect{} employs a sampling mechanism using the timer mechanism. When the timer is expired, \SheriffDetect{} tracks memory writes in the current period using the twinning and diffing mechanism. The detailed mechanism is described in Figure~\ref{fig:sheriffdetect}. 

To do this, \SheriffDetect{} also creates a ``temporary twin'' page for every page that have been accessed when the sampling timer is expired. Because these ``temporary twin'' pages are thread-private, we can create and update these pages in the timer handler by simply copying from their corresponding ``working'' pages. The difference between a ``working'' page and its ``temporary'' page implies at least one memory write in this sampling period. Currently, \SheriffDetect{} samples memory accesses of each thread at every 10 microsecond. The tradeoff between sampling period and performance is also discussed in Section~\ref{}. 

\subsubsection{Tracking Cache Invalidations}
\label{sec:invalidation}
As the discussion in Section~\ref{sec:detectionidea}, \SheriffDetect{} tracks and reports those cache lines with a big number of cache invalidations, where they are considered to cause serious performance problems. 

In order to track cache invalidations, \SheriffDetect{} introduces a cache line status word for every cache line of the globals and heap, showed in Figure~\ref{fig:cachelinestatusword}.  Every cache line status word contains two fields, the last thread writing to this cache line and the number of cache invalidations of this cache line. 
Every time, when \SheriffDetect{} detects a memory write, either at the end of transactions or in the sampling timer handler,  it updates these two fields correspondingly. Based on the assumptions described in Section~\ref{sec:detectionidea}, \SheriffDetect{} increments the number of cache invalidations when there is a write from a different thread. To avoid using lock, \SheriffDetect{} updates those counters using atomic primitives. 

\subsection{Optimizations}

\SheriffDetect{} also employs the following optimizations in order to reduce the performance overhead. 

\paragraph{Getting Callsite Information.}
\SheriffDetect{} intercepts memory allocation operations for collecting callsites for every heap object. To reduce the performance overhead, \SheriffDetect{} do not use the bracktrace(), but identify the callsite by analyzing return or frame address using GCC extensions. However, this can not work on applications without debugging information. 

\paragraph{Reducing timer overhead.}
As explained in Section~\ref{sec:memoryaccesses}, \SheriffDetect{} uses sampling to track cache invalidations. To reduce the impact of timer interrupts, \SheriffDetect{} activates sampling only when the average transaction time is larger than a threshold time (currently 10 milliseconds). \SheriffDetect{} uses an exponential moving average to track average transaction times ($\alpha = 0.9$). This optimization does not significantly reduce the possibility of finding false sharing since \SheriffDetect{}'s goal  is to find an object with a large amount of interleaved writes from different threads.

\paragraph{Sampling to find shared pages.} 
To reduce this overhead, \SheriffDetect{} leverages a
simple insight: if two threads can falsely share a cache line,
then they must simultaneously access the page containing
that cache line. \SheriffDetect{} relies on page protection to determine whether pages are shared or not. When one application has a large number of transactions or page touches, the protection overhead to gather this sharing information can dominate running time.

\SheriffDetect{} reduces overhead by using sampling to detect shared pages. If objects on a page are frequently falsely shared, the page itself must also be frequently shared, so even relatively infrequent sampling will eventually detect this sharing.  \SheriffDetect{} currently samples the first 50 out of every 1,000 periods (one period equals one transaction or one sampling interval). At the beginning of each sampled period, all memory pages are made read-only so that any
writes to each page will be detected. Upon finding a page is
shared, \SheriffDetect{} will track any false sharing inside it. \SheriffDetect{} only updates the shared status of pages during sampled periods and at commit points. During unsampled periods, pages whose sharing status is unknown impose no protection overhead.

\subsection{Limitation}
\label{discussion:faultofdetect}

Unlike previous tools, \SheriffDetect{} has no false positives, differentiates true sharing from false sharing, and avoid false positives caused by the reuse of heap objects.
\SheriffDetect{} can under-report false sharing instances in the following situations:

\paragraph{Single writer.}
False sharing usually involves updates from multiple threads, but it can also arise when there is exactly one thread writing to part of a cache line while other threads read from it. Because its detection algorithm depends on at least one differing update (that is, at least two writes of distinct values), \SheriffDetect{} cannot detect this kind of false sharing (though \sheriffprotect{} eliminates it; see Section~\ref{sec:patrol}).

\paragraph{Heap-induced false sharing.}  
\sheriff{} replaces the standard memory allocator with one that, like the Hoard allocator, avoids most heap-induced false sharing. \sheriff{}'s memory allocator (like Hoard), carves memory into page-sized chunks; each thread allocates
from its own set of chunks, and the allocator never splits cache lines across threads. Because \SheriffDetect{} uses the same allocator, it cannot detect false sharing that would be caused by the standard memory allocator. Since it is straightforward to deploy Hoard or a similar allocator to avoid heap-induced false sharing, this limitation is not a problem in practice.

\paragraph{Misses due to sampling.}  Since it uses sampling to
  capture continuous writes from different threads, \SheriffDetect{} can miss writes that occur in the middle of sampling intervals. We hypothesize that false sharing instances that affect performance are unlikely to perform frequent writes exclusively during that time, and so are unlikely to be missed.


\section{Tolerating False Sharing}
\input{sheriff/prevention}

\section{Experimental Evaluation}
\label{sec:evaluation}

We perform our evaluations on a quiescent 8-core system (dual
processor with 4 cores), with 8GB of RAM. Each processor is a 4-core 64-bit Intel Xeon, running at 2.33 Ghz with a 4MB L2 cache. For compatibility reasons, we compiled all applications to a 32-bit target using GCC. All performance data is the average of 10 runs, excluding the maximum and minimum values.

The evaluation answers the following questions:

\begin{itemize}
\item How effective is \sheriffdetect{} at finding false sharing and guiding programmers to their resolution? (Section~\ref{sec:effecteval})
\item What is \sheriffdetect{}'s performance overhead? (Section~\ref{})
\item How sensitive is \sheriffdetect{} to different sampling rate? (Section~\ref{}) 
\item How effective does \sheriffprotect{} mitigate false sharing? (Section~\ref{})
\end{itemize}

\subsection{\sheriffdetect{} Effectiveness}

\label{sec:effecteval}

This section evaluates whether \sheriffdetect{} can be used to find false sharing problems, both in synthetic test cases and in actual applications.

We developed a range of microbenchmarks that exemplify different situations related to false sharing. We evaluate these benchmarks on both \SheriffDetect{} and Intel's Performance Tuning Utility(PTU v3.2), the previous state-of-the-art work of false sharing detection.
Detection results are shown in Table~\ref{table:microbenchmarks}. \sheriffdetect{} only reports those false sharing instances that can possibly affect the performance, while correctly ignores those cases without no performance impact.
PTU has false alarms/positives.  It does not track those pattern of accesses, which reports false positives for those non-interleaved accesses. Also, PTU does not track memory deallocations, thus it can not filter out those pseudo false sharing caused by memory reuse. \sheriffdetect{} avoids all of these problems and reports false sharing problems correctly. 


\begin{table}
\centering
\begin{tabular}{l|l|l|l}
\hline
{\bf \small Microbenchmark} & {\bf \small Perf Sensitive } & {\bf \small \sheriffdetect{} } & {\bf \small PTU } \\
\hline

\small \textbf{False Sharing (adjacent objects)} & YES & \cmark{} & \cmark{} \\
\small \textbf{False Sharing (same object)} & YES & \cmark{} & \cmark{} \\
\hline
\small \textbf{True Sharing} & NO & & \\
\small \textbf{Non-interleaved False Sharing} & NO & & \xmark{}\\
\small \textbf{Heap Reuse(no sharing)} & NO & & \xmark{}\\
\hline
\end{tabular}
\caption{False sharing detection results using PTU and \sheriffdetect{}. \sheriffdetect{} correctly reports only actual false sharing instances, with a performance impact;
\cmark{} indicates a correct report and \xmark{} indicates a false alarm. 
\label{table:microbenchmarks}}
\end{table}

We further evaluate \SheriffDetect{} and PTU on two widely-used benchmarks suites, Phoenix~\cite{phoenix-hpca} and PARSEC~\cite{parsec}. We use the simlarge inputs for all applications of PARSEC. For Phoenix, we chose available parameters that allow the programs to run as long as possible. As of this writing, we were unable to successfully
compile \texttt{raytrace} and \texttt{vips}, and \sheriff{} is
currently unable to run \texttt{x264}, \texttt{bodytrack},
and \texttt{facesim}. Freqine currently can not support pthreads. Thus, those benchmarks are excluded here. 
 
\begin{table}
\centering
\begin{tabular}{l|r|r}
\hline
{\bf \small Benchmark} & {\bf \small PTU} & {\bf \small \sheriffdetect{}}\\
 & {\# Lines} & {\# Objects}\\
\hline
\small \textbf{kmeans} & 1916 &  2 \\
\small \textbf{linear\_regression} & 5 & 1 \\
\small \textbf{matrix\_multiply} & 468 & 0\\
\small \textbf{pca} & 45 & 0 \\
\small \textbf{reverseindex} & N/A & 5 \\
\small \textbf{word\_count} & 4 & 3\\
\hline
\small \textbf{canneal} & 1 & 1 \\
\small \textbf{fluidanimate} & 3 & 1 \\
\small \textbf{streamcluster} & 9 & 1\\
\small \textbf{swaptions} & 196 & 0\\
\hline
\small \textbf{Total} & 2647 & 14\\
\hline
\end{tabular}
\caption{Overall detection results of PTU and \sheriffdetect{} on Phoenix and PARSEC benchmark suites. We only list those benchmarks that at least one of tools reports false sharing problems. For PTU, we show how many cache lines are marked as falsely shared. For \sheriffdetect{}, we show how many objects are reported by \sheriffdetect{} (with cache invalidations larger than 100). The item marked as ``N/A'' means PTU fails to show results because it runs out of memory.
\label{table:fsdetection}}
\end{table}


The overall results are shown in Table~\ref{table:fsdetection}. PTU reports that 2647 cache lines may exist false sharing problems, given that they can report false positives. \sheriffdetect{} reveals that seven out of sixteen evaluated benchmarks have some false sharing issues. Totally, only 14 objects are reported, but only 4 of them shows a big number of cache invalidations. 

Several reasons contributes to this big difference. First, PTU reports cache lines information about false sharing objects, while \SheriffDetect{} only reports objects. Second, PTU reports multiple times if a heap object, with the same allocation site, is allocated multiple times. 
Third, PTU may report false positives since it do not track interleaved accesses and overrate the problems caused by heap reuses. 


We manually fix these four false sharing problems based on reports of \SheriffDetect{}, and showed the performance data in Table~\ref{table:perfafterfix}. To explain why performance improvement are different, we examine the maximum possible updates occurred on these false sharing objects, the reason of performance improvement. For example, \texttt{linear\_regression} has the largest updates, thus causes serious performance problem because of false sharing. 

\begin{table}
\centering
\begin{tabular}{l|r|r}
\hline
{\bf \small Benchmark} & {\bf \small Performance Improvement} & {\bf \small Updates}\\
 & & (M)\\
\hline
\small \textbf{linear\_regression} & 818\% & 1323.6\\
\small \textbf{reverseindex} &  2.4\% & 0.4\\
\small \textbf{streamcluster} & 5.4\% & 28.7\\
\small \textbf{word\_count} &  1\% & 0.3\\
\hline
\end{tabular}
\caption{Performance data for four false sharing benchmarks. 
All data are obtained using the standard \pthreads{} library. 
``Updates'' shows how many million updates (in total) occurred on falsely-shared cache lines.
\label{table:perfafterfix}}
\end{table}


In \texttt{reverse\_index} and \texttt{word\_count}, multiple threads repeatedly modify the same heap object. The pseudo code for these two benchmarks are listed in Figure~\ref{fig:reverseindex}. We may use thread-local copy to avoid the false sharing problem here; each thread can modify a temporary variable first and then modify the global \texttt{use\_len} in the end of thread.

\begin{figure}[!t]
\begin{lstlisting}
int * use_len;
void insert_sorted(int curr_thread) {
   ......	
   // After finding a new link
   (use_len[curr_thread])++;
   ......	
}
\end{lstlisting}
\caption{A fragment of source code from \texttt{reverse\_index}. False sharing arises when adjacent threads 
modify the \texttt{use\_len} array. 
\label{fig:reverseindex}}
\end{figure}

\texttt{Linear\_regression}'s false sharing problem is a little different (see Figure~\ref{fig:linear_regression}). 
Two different threads write to the same cache line when the
structure \texttt{lreg\_args} is not cache line aligned. This problem can be avoided easily by padding the structure \texttt{lreg\_args}.

\begin{figure}[!t]
\begin{lstlisting}
struct {
  long long SX;
  long long SY;
  long long SXX;
  ......
} lreg_args;

void *lreg_thread(void *args_in) {
  struct lreg_args * args = args_in;
  for(i = 0; i < args->num_elems; i++) {
    args->SX  += args->points[i].x;
    args->SXX += args->points[i].x 
   	         * args->points[i].x;
  }
  ......	
}
\end{lstlisting}
\caption{A fragment from \texttt{linear\_regression} code. 
Each thread is passed in a different address (\texttt{struct lreg\_args}) and each thread can work on its corresponding \texttt{args\_in}. 
Unfortunately, the size of \texttt{struct lreg\_args} is not cache line aligned (52 bytes) and that
causes two different threads to write to the same cache line simultaneously. 
\label{fig:linear_regression}}
\end{figure}

The false sharing problem detected in \texttt{streamcluster} (one of the PARSEC benchmarks) is similar to the false sharing problem in \texttt{linear\_regression}; two different threads are writing on the same cache line.  In fact, the author tried to avoid the false sharing problems and make every stride a multiple times of cache line size. But the default cache line size is 32 bytes, which is different from the actual physical cache line size that we are used in evaluation (64 bytes).  By simply setting the \texttt{CACHE\_LINE} macro to 64 bytes, it is possible to avoid this false sharing problem completely.


\subsection{Ease of locating false sharing problems}

\noindent
To illustrate how \sheriffdetect{} can precisely locate false sharing problems, we 
use one benchmark (\texttt{word\_count}, a PHOENIX benchmark) as
an example. Our experience with diagnosing other false sharing issues is similar.

Here is an example output from \sheriffdetect{} from \texttt{word\_count}.

\begin{verbatim} 
1st object, cache interleaving writes 
13767 times (start at 0xd5c8e140). 
Object start 0xd5c8e160, length 32. 
It is a heap object with callsite:
callsite0:./wordcount_pthreads.c:136
callsite1:./wordcount_pthreads.c:441
\end{verbatim}

Line 136 (\texttt{wordcount\_\pthreads{}.c}), 
contains the following memory allocation call:

\begin{verbatim}
use_len=malloc(num_procs*sizeof(int));
\end{verbatim}

Grepping for \texttt{use\_len}, a global pointer, quickly leads to this line:

\begin{verbatim}
use_len[thread_num]++;
\end{verbatim}

Now it is clear that different threads are modifying the same object
(use\_len). Fixing the problem by using the
thread-local data copies is now straightforward~\cite{detect:intel}.

By contrast, compare PTU's output in Figure~\ref{fig:wordcount}. Finding this problem is far more complicated with PTU, since it only presents functions using each cache line, not to mention the fact that PTU can
report huge numbers of false positives.  Another shortcoming
of PTU is that ``Collected Data Refs'' number cannot be used as a metric to evaluate the significance of false sharing problems. For this example, PTU only reports 12 references (versus 13767 times for \sheriffdetect{}).

\begin{figure*}[!t]
\centering
\includegraphics[width=6in]{sheriff/figure/wordcount}
\caption{PTU output for \texttt{word\_count}.
\label{fig:wordcount}}
\end{figure*}

%That is why we cannot relying on PTU to do the analysis of false sharing
%problems given the large number of cache lines involved. 

\subsection{\sheriffdetect{} Performance Overhead}
\label{sec:results-runtime-overhead}

\begin{figure*}[!t]
\centering
\includegraphics[width=6in]{sheriff/figure/detective.png}
\caption{\sheriffdetect{} overhead across two suites of benchmarks,
  normalized to the performance of the \pthreads{} library (lower is better). 
  With two exceptions, its overhead is acceptably low.
\label{fig:overhead}}
\end{figure*}


This section shows the runtime overhead of \sheriffdetect{} (comparing to \pthreads{})on
two multithreaded benchmarks suites, PHOENIX and PARSEC.  The results
can be seen from Figure~\ref{fig:overhead}.  

\texttt{linear\_regression} exhibits almost
a 10X speedup against the one using \pthreads{} library even with the added overhead of sampling and 
other mechanisms of \sheriffdetect{}.  There is a
serious false sharing problem inside (see
Table~\ref{table:perfafterfix}) which both \sheriffdetect{} and \sheriffprotect{} eliminate
automatically. 

There are two benchmarks on which \sheriffdetect{} do not perform well. 
One is \texttt{canneal}, the performance overhead of \sheriffdetect{}
on this benchmark is about 7X slower than the one using \pthreads{}
library. Another one is \texttt{fluidanimate}, the performance overhead is about 
14X slower than that using \pthreads{}.

According to our analysis, the transaction number and dirty pages are two main causes 
of the overhead. For most of time, more transaction number can cause more dirty pages.  
In order to find out what can affect the performance of these two benchmarks, 
we get some characteristics data(see Table~\ref{tbl:characteristics} about these 
two benchmarks when they are using 
the \sheriffdetect{}.

\begin{table}
\centering
\begin{tabular}{|l|r|r|r|}
\hline
{\bf \small Benchmark} & {\bf \small Trans} &{\bf \small DirtyPages} & {\bf \small Runtime} \\
 & {\#} & {M} & {s}\\
\hline
{\bf \small canneal} & 930 & 2.9 & 74.3 \\
{\bf \small fluidanimate} & 18696114 & 2.15 & 21.56\\
\hline
\end{tabular}
\caption{Characteristics of slower benchmarks in \sheriffdetect{}.
\label{table:characteristics}}
\end{table}

From the table, we can easily find out that these two benchmarks share the same attribute, having large amount of dirty pages. 
For one dirty page, \sheriffdetect{} need two protections, creation of the Copy-On-Write version and
different version of twin pages, checking the false sharing problem inside every periodical checking cycle and 
commits to the shared mapping. Given large amount of dirty pages, copying alone is very expensive 
since one dirty page needs at least 3 copies. 
For \texttt{fluidanimate}, 2.2 million pages needs about 6.8 million copies, which can 
acount for about 20 seconds copying overhead since copying one gigabyte of memory takes approximates 0.75 seconds.
shared pages, which leads to substantial overhead.
Examination of the source code of \texttt{fluidanimate} reveals a large number of lock
calls, \sheriff{} replaces lock calls with their interprocess variants
and triggers a transaction end and begin for each, adding some overhead if there are some shared pages.
 
\begin{comment}
The worst case for \sheriff{} is exemplified
by \texttt{ferret}, which modifies a huge number of pages (about
3.45G) and has a large number of transactions (about 1M).
We also measured charecteristics of our benchmark suites in
Table~\ref{table:characteristics}.  The
following parameters determine the performance of \sheriff{}.

\begin{itemize}
\item
Pages written: each write on a protected page imposes
additional overhead to unprotect the page in the page fault handler.
In the sampling handler, \sheriff{} must check for cache writes for
each shared written page, and at the end of transaction, \sheriff{} must
check cache writes for each page and commit the modification to the shared
space.

\item
Transaction length: \sheriff{} introduces overhead in the beginning
of transaction and in the end of each transaction. Longer transactions
amortizes this overhead.

\item 
Allocation times: \sheriff{} (in detection mode) attaches callsite information for every
allocated object, slowing allocation.

\item
Cache cleanup size: \sheriff{} cleans up the invalid cache counting
information in the memory allocation if one allocation is involving in
the re-usage of memory of those freed memory objects.
\end{itemize}

From the results from Table~\ref{table:characteristics}, we can
confirm our analysis.  Allocation times and cache cleanup size have
little impact on performance. However, when the number and rate of
pages written is large, performance suffers.

Figure~\ref{fig:overhead} shows that \sheriff{}'s overhead is highest for
the following two benchmarks: \texttt{fluidanimate} and \texttt{canneal}.
For \textt{canneal}, different threads are writting to a lot of shared pages
benchmarks \texttt{ferret}, \texttt{reverse\_index}, \texttt{dedup}
and \texttt{fluidanimate}. Characteristics showed in
Table~\ref{table:characteristics} that the first three benchmarks have
a very high rate of page updates (\textbf{PagesPerMs}). 
\texttt{fluidanimate} is an outlier if we are just using the \textbf{PagesPerMs} metrics.
The reason of \texttt{fluidanimate} has a high overhead is that there
are huge amounts of transactions inside (about 10M). Examination of
the source code revealed a large number of lock calls in this
application. \sheriff{} replaces lock calls with their interprocess
variants and triggers a transaction end and begin for each, adding
overhead.  The worst case for \sheriff{} is exemplified
by \texttt{ferret}, which modifies a huge number of pages (about
3.45G) and has a large number of transactions (about 1M).

\begin{table*}
\centering
\begin{tabular}{|l|rrrr|rr|r|}
\hline
{\bf \small Benchmark} & {\bf \small PagesWritten} & {\bf \small Commits} & {\bf Allocs} & {\bf \small CleanupSize} & {\bf \small TranLength(ms)} & {\bf \small PagesPerTran} & {\bf \small PagesPerMs}\\
\hline
\small \textbf{histogram} & 0 & 24 & 2 & 0 & 12.5 & 0 & 0\\
\small \textbf{kmeans} & 1312 & 3836 & 101002 & 0 & 4.15 & 0.34 & 0.08\\
\small \textbf{linear\_regression} & 16 & 24 & 3 & 0 & 38.6 & 0.67 & 0.02\\
\small \textbf{matrix\_multiply} & 16 & 24 & 11 & 0 & 313.23 & 0.67 & 0.0\\
\small \textbf{pca} & 0 & 47 & 2 & 0 & 450.69 & 0 & 0.0\\
\small \textbf{reverseindex} & 260201 & 156409 & 250927 & 0 & 0.05 & 1.66 & 30.99 \\
\small \textbf{string\_match} & 0 & 24 & 7 & 0 & 104.75 & 0 & 0.00\\
\small \textbf{word\_count} & 145 & 89 & 38 & 32 & 25.08 & 1.63 & 0.06\\
\hline
\small \textbf{blackscholes} & 0 & 23 & 4 & 0 & 453.51 & 0 & 0.0\\
\small \textbf{canneal} & 8 & 1056 & 5974612 & 0 & 10.32 & 0.01 & 0.0\\
\small \textbf{dedup} & 76184 & 45636 & 8291 & 0 & 0.04 & 1.67 & 44.9\\
\small \textbf{ferret} & 904381 & 1072258 & 110558 & 0 & 0.01 & 0.84 & 76.04\\
\small \textbf{fluidanimate} & 8 & 10018550 & 135430 & 352 & 0.00 & 0.00 & 0.00\\
\small \textbf{freqmine} & 0 & 1 & 33 & 0 & 11524.6 & 0 & 0.0 \\
\small \textbf{streamcluster} & 32824 & 128557 & 12 & 294 & 0.02 & 0.26 & 10.42\\
\small \textbf{swaptions} & 48 & 24 & 388 & 0 & 167.23 & 2 & 0.01\\
\hline
\end{tabular}
\caption{Characteristics of benchmarks. 
\label{table:characteristics}}
\end{table*}
\end{comment}
%%%%%%%%%%%%%%%%%%%%%%%%%%%%%%%%%%%%%%%5
%%%% Some data to list the effectiveness of this tool.
%%%%%% How many caches are carried for each test case. 
%%%%%% Whether all caches has false sharing problem.
%%%%%%%%%%%%%%%%%%%%%%%%%%%%%%%%%%%%%%%
\subsection{Performance of \sheriffprotect{}}
\label{sec:results-runtime-overhead}

\begin{figure*}[!t]
\centering
\includegraphics[width=6in]{sheriff/figure/patrolperf.pdf}
\caption{\sheriffprotect{} performance across two suites of benchmarks,
  normalized to the performance of the \pthreads{} library (see
  Section~\ref{sec:results-runtime-overhead}). In case of
  catastrophic false sharing, \sheriffdetect{} dramatically increases performance.
\label{fig:patrol}}
\end{figure*}

Here, we examine the performance improvement by tolerating the false sharing problems in
\sheriffprotect{}.
The performance improvement can be seen in the Figure~\ref{fig:patrol}.  

From the results, we can see that \texttt{linear\_regression} exhibits almost
a 10X speedup against the one using \pthreads{} library.  
By tolerating the serious false sharing problem inside (see
Table~\ref{table:perfafterfix}), \sheriffprotect{} achieves a significant performance 
benefit for this benchmark.
\texttt{histogram} performance benefit comes from one munmap() call to unmap about 400M's file, 
we currently are not sure about why multi-process framework can perform better in this case.

There are three benchmarks which runs at most 30\% slower than using the \pthreads{}. 
We examine the reasons to cause this slowdown. 
For \texttt{kmeans}, this application creates more than 3000 threads about 8 seconds. Since the overhead
to create one process is higher than that to create one thread, this part of 
overhead dominates most of overhead. 

For \texttt{reverse\_index} and \texttt{fluidanimate}, 
they exhibit slowdown because of the use of the processes-as-threads framework. 
This performance impact arises
from the use of a file-based mapping, which connects the private
mapping and shared mapping. The Linux page fault handler does more
work when operating on file-based pages than on anonymous pages (the
normal status of heap-allocated pages). The first write on a
file-mapped page repopulates information from the file's page table
entry. Also, the shared store for all heap pages is initially set to
\texttt{MAP\_SHARED}, so writing to one shared page can cause a
Copy-On-Write operation in the kernel even when there is only one user.
\texttt{fluidanimate} has an enormous number of transactions(18 Million), \sheriffprotect{} 
introduces some additional ovherhead for every trnasaction. That also accounts for part of
overhead.

