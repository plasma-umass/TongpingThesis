\date{\vspace*{-0.2in}}


\documentclass[10pt]{sigplanconf}
\nocaptionrule


\newcommand{\footnotenonumber}[1]{{\def\thempfn{}\footnotetext{\small #1}}}
\usepackage[normalem]{ulem}
\usepackage{graphicx}
\usepackage{times}
\usepackage{subfigure}
\usepackage{url}
\urlstyle{rm}

\usepackage{color}
\usepackage{listings}
\usepackage{amsmath}
\usepackage{amsfonts}
\usepackage{amssymb}
\usepackage{comment} 

\newtheorem{thm}{Theorem}
\newtheorem{prop}[thm]{Proposition}
\newtheorem{cor}[thm]{Corollary}
\newtheorem{lem}[thm]{Lemma}
\newtheorem{defn}[thm]{Definition}

\newcommand{\cfunction}[1]{{\bf \tt #1}}
\newcommand{\malloc}{\cfunction{malloc}}
\newcommand{\realloc}{\cfunction{realloc}}
\newcommand{\free}{\cfunction{free}}
\newcommand{\madvise}{\cfunction{madvise}}
\newcommand{\brk}{\cfunction{brk}}
\newcommand{\sbrk}{\cfunction{sbrk}}
\newcommand{\mmap}{\cfunction{mmap}}
\newcommand{\munmap}{\cfunction{munmap}}
\newcommand{\mprotect}{\cfunction{mprotect}}
\newcommand{\mlock}{\cfunction{mlock}}

\hyphenation{app-li-ca-tion}
\hyphenation{Die-Hard}
\hyphenation{Archi-pe-la-go}
\hyphenation{buf-fer}

\lstset{language=c++, basicstyle=\small\ttfamily,frame=single,tabsize=4}

\definecolor{Gray}{cmyk}{0,0,0,0.5}

\begin{document}
\conferenceinfo{XXXXXXXXXXXXXXXXXXX}

\title{\Large \bf Sheriff: Precisely Pinpoint False Sharing Objects in Multi-threaded Programs}

\authorinfo{Tongping~Liu \and Emery~D.~Berger}
{Dept.\ of Computer Science \\ University of Massachusetts, Amherst \\
Amherst, MA 01003} {\{tonyliu,emery\}@cs.umass.edu}

\maketitle

\begin{comment}
%%%%%%%%%%%%%%%%%%%%%%%%%%%%%%%%%%%%%%%%%%%%%%%%%%%%%%%%%%%%%%%%%%%%%%%%%%%%%%%%%%%%%%%%%%%%%
Story:
Multi-core processors or NUMA are now widely used in order to avoid the physical limits of hardware. 
Multi-threaded is considered as one way to take full advantage of those computation resources. Unfornately, 
writing efficient multi-threaded program is not an easy task; 
false sharing problem can reduce the performance greatly, 
even worse, one program with serious false sharing problem can run slower on multi-core machine that that on 
on single-core machine with the same cpu frequency of every core.

This paper presents Sheriff, a system to detect those false sharing problem in multi-threaded c/c++ programs. 
Comparing to previous tool, Sheriff has a very low performance overhead, 
averagely the performance overhead is about 15\% in our experiments. 
For most programs, Sheriff can run almost the same speed or even faster than original program
using pthreads library. 
The second advantage of Sheriff is that there is no false positives at all. 
The false sharing problems reported by Sheriff are actual false sharing problems.

The third adanvatage of Sheriff is that Sheriff can pinpoint the code to cause the false sharing problems. 
For heap objects, Sheriff can point out the allocation site, 
then it is easy for programmer to fix the false sharing problem given the callsite information,
even for some one that are not familiar with the code. 
For global objects, Sheriff can point out those objects' name and length information which has false sharing problem.

How Sheriff to do that?
First, we are using a runtime system which maintains the semantics of multi-threaded program.
To detect those modifications by different threads, we turn those multi-threaded program into a 
multi-process program and use the page protection mechanism to capture those writes on different threads.
In order to capture those writes of different memory, we introduce a "twin" page in the page fault handler;
then we utilize the lazy differentiating mechanism (which has been widely used in distributed share memory) to 
find acutal writes in each phase. 
In order to capture those cache invalidation of different threads, we maintain a global array which is used to
maintain the last thread id to write on one cache line. We use a 
conservative mechanism to count those invalidation of cache lines, 
only those interleaving writes by different threads are counted as an invalidation of corresponding cache line. 
In order to differentiate the false sharing problem from true sharing problem, 
we also keep a word-based version number array 
which is used to record every writes information on each word, including thread id and version. (We use a special
id to identify multi-threads on one word).
 
Also, this word-based versioning mechanism
can help us to the precisely locate the object that are causing the problem if multiple objects are located in 
the same physical cache line.

In order to catch those false sharing problem caused by long transaction, 
we introduce one "sampling mechanism" which can get continutive writes of different threads.

In order to pingpoint the allocation site, we will attach the callsite information for those allocation. Then we can present those callsite when we find out much invalidation. 
 
What else Sheriff can do?
First, Sheriff can work as a run-time system, which can tolerate some of the false sharing problems. 
Second, Sheriff can work as a system which can detect those race condition problems undetectable using those lock-set based tool.
Third, Sheriff can also work as a system to tolerate race problem, like Isolator.

Future work:
(1) Pinpoint the line number to access the cache line by using the "watch point" technique.
(2) Figure out the problem caused by read-write false sharing problem by using the "watch point" technique.
(3) Design a run-time system which can tolerate the false sharing problem in a very low overhead. Profiling 
on specified input should be very helpful to find the problem.
%%%%%%%%%%%%%%%%%%%%%%%%%%%%%%%%%%%%%%%%%%%%%%%%%%%%%%%%%%%%%%%%%%%%%%%%%%%%%%%%%%%%%%%%%%%%%
\end{comment}

\begin{abstract}
\end{abstract}

\terms
Performance, False sharing

\keywords
Concurrency, False Sharing, Performance, Multi-threaded program

%%%%%%%%%%%%%%%%%%%%%%%%%%%%%%%%%%%%%%%%%%%%%%%%%%%%%%%%%%%%%%%%%%%%%%%%%%%%%%%%%%%%%%%%%%%%%
%%%%%%%%%%%%%%%%%%%%%%%%%%%%%%%%%%%%%%%%%%%%%%%%%%%%%%%%%%%%%%%%%%%%%%%%%%%%%%%%%%%%%%%%%%%%%

\section{Introduction}

The false sharing problems can affect the performance greately, detection of false sharing is very important? \\
Definition of false sharing\\
How is state of art? Some approaches try to avoid that(compilation, scheduling, heap allocator), but none of them 
can avoid all false sharing problems successfully. \\
That is why we need the detection tool. What is the shortcomings of current detection tool, performance, false positives, far from the problems \\
What is the contribution of Sheriff? one runtime system to simulate multi-threaded program, low overhead, no false positives, precisely locate the root of problem\\
What is the target of Sheriff? How many types of false sharing Sheriff can detect? \\ 
What is the organization of this paper? \\
%Another good part of this paper is that we can avoid the problem caused by heap objects as much as possible. For example, we 
% can try to allocate the object in the same place and we can cleanup the data for heap objects. 

\section{Sheriff Overview}

How Sheriff can be used to capture those false sharing problem? Assumption of Sheriff? 
%Assumption: those programs to be checked is right in lock usage, without deadlock, race and atomicity violation problems.
What is the basic mechanism of Sheriff? 
\subsection{Observation}
\subsection{Targets}
\subsection{Procsses as Threads}
\subsection{Twin Mechanism}

\section{Simulating multi-threaded program}
We are going to answer the following questions in the following subsections:
\begin{itemize}
\item How to turn threads into proceses?
\item How to maintain the semantics to share address space?
\item How to share file access?
\item How to handle synchronization?
\end{itemize}

\subsection{Thread Creation and Exit}
\subsection{Share Memory}
\subsection{Share File Access}
\subsection{Synchronization}
mutex, barrier, condvar
\subsection{Thread Execution}
\subsubsection{Transaction Begin}
\subsubsection{Execution}
\subsubsection{Transaction End}

\section{Detection of False Sharing}
In this section, we are going to talk how to utilize the runtime system to locate the false sharing objects.
This section are trying to answer the following questions:

\begin{itemize}
\item How to capture the memory access from different process?
\item How to capture the continuous memory accessing?
% - sampling mechanism - "temporary twin" and "original twin". 
\item How to capture the interleaving cache invalidation? 
% Use a global array, updating timely when modification is detected. 
\item How to identify objects inside one cache line? 
%Attach the callsite information to capture the allocate sites for heap objects. 
\item How to differentiate true sharing and false sharing?
%detect the combination? An array to get word version and threads working on that. We can detect those fields inside one object causing the problem too.
\item How to report?
% In the end of program, we traverse the whole global array. 
\end{itemize}
\subsection{Capture of Memory Writes}
\subsection{Capture of Continuous Writes}
\subsection{Capture of Interleaving Cache Invalidation}
\subsection{Indentify Objects inside Cache Line}
\subsection{Avoidance of False Positives}
\subsection{Reporting False Sharing Objects}

\section{Discussion}
\subsection{Performance Improvment}
\begin{itemize} 
\item Protecting small objects only.
\item Remove protection if only one thread is running.
\item Page version number to improve performance.
\end{itemize} 

\subsection{Limitation}
It is not good to use stack to communicate between threads, Sheriff cannot run that type of program sucessfully.

\section{Evaluation}
\begin{itemize}
\item Effectiveness of tool
\item What is the performance of tool? 
\item What can affect the performance of tool, list some data(commits, page protection).
%% We will try several test cases, something like simple false sharing, combination of false sharing and 
%% true sharing. 
%% True sharing benchmarks, to see how Sheriff works in these cases.
  
\end{itemize}
\subsection{Effectiveness of Sheriff}
\subsubsection{Micro-benchmarks}
\subsubsection{Actual Applications}
\subsection{Performance of Sheriff}
%Whether we can just run Sheriff on a sigle-core machine to capture the false sharing problems? Yes.

\section{Related Work}
\label{sec:relatework}
\begin{itemize} 
\item Tool to detect false sharing problem of multithreaded program.
\item Distributed Share Memory : twin page usage.
\end{itemize}

\section{Future Work}
\begin{itemize}
\item Pinpoint the line number to access the cache line by using the "watch point" technique.
\item Figure out the problem caused by read-write false sharing problem by using the "watch point" technique. 
\item Design a run-time system which can tolerate the false sharing problem in a very low overhead. 
\end{itemize}

\label{sec:future}


\section{Conclusion}
\label{sec:discussion}

\section{Acknowledgement}
\section{referrence}
%\input{ref}
\end{document}
