\label{sec:applications}

\doubletake{} implements three important applications based on its lightweight 
dynamic analysis framework. 
Those applications are implemented in a best-effort way to show the efficiency of our framework.
They are not targeted for a complete and noval solution for these problems, and most of
mechanisms are borrowed from previous work. 

For these applications, \doubletake{} runs a program at full speed during an epoch and 
finds evidence of possible memory problems in the end of each epoch. 
If \doubletake{} detects memory problems, it rolls back the program and re-execute it 
in order to collect detailed information problems and report to users. 
Different applications has different implementations, which are discussed in more detailed 
in the following. 

\subsection{Detection of Heap Buffer Overflows}
\label{sec:overflow}
Buffer overflows can greatly impact programs' robustness and security. When a program has 
buffer overflows, a program can exit unexpectedly with segmentation fault errors or other errors. 
Even worse, overflows can be maliciously exploited, loosing security and privacy guarantees. 
Buffer overflows are still one of dominant errors even after decades of research in this field.
In 2012, four types of the CWE/SANS "Top 25 Most Dangerous Software Errors" and 
over one-third valnuerabilites of 1818 high serverity problems reported from 
NIST are related to buffer overflows~\cite{overflows1, overflows2, overflow:lbc}. 
Among them, a big signifiant of problems are heap buffer overflows. 

\subsubsection{Detection}
To detect heap buffer overflows, \DoubleTake{} puts canaries before and after actual heap 
objects in normal executions ~\cite{overflow:lbc, AddressSanitizer}. 
At the end of each epoch, \doubletake{} verifies integrities of canaries by traversing 
the bitmap, which marks the placement of canaries and discussed in Section~\ref{sec:canaries}.  
Corrupted canaries indicates heap buffer overflows. 
To detecting those one-byte overflows, \doubletake{} embeds actual size information of objects 
into headers of objects and places byte-based canaries for non-aligned objects.

\subsubsection{Reporting}
\label{sec:overflowreport}
When a program is found to have heap overflows, \doubletake{} rolls back the execution 
and re-executes this program.
To precisely identifying the instruction reponsible for buffer overflows, 
\doubletake{} installs watch points at the address of corrupted canary before re-execution, 
which is discussed in Section~\ref{sec:watchpoints}.
When the program is re-executed, any instruction that writes to this address 
will trigger a debug trap (resulting in a SIGTRAP signal). 
\doubletake{} then reports callsite information of trapped instructions. 
This can be used to pinpoint the
exact source location where the buffer overflow occurred.
  
\subsection{Detection of Memory Leakage}
\label{sec:leak}
Memory usage inefficiency can greatly reduce programs performance and availability. 
A system with memory leaky programs can greatly degrade reponsiveness 
if some programs has to be paged out, caused by excessive memory consumption by leaky programs. 
Memory leaks grows memory consumption over time, which can make a program runing slower 
and slower, or even unreponsively in the end. 
Memory leakage is still one of common classes of reported bugs.

\subsubsection{Detection}
\doubletake{} detects memory leakage using the same marking mechanism as 
conservative garbage collection ~\cite{Wilson:1992:UGC:645648.664824}.
Starting from roots (globals, stack, and registers), \doubletake{}
computes all reachable heap objects. Any unreachable object that
has not been freed must be leaked.

More specifically, \doubletake{} checks all values of globals, stack and registers. 
If a value falls into the range of the allocated heap, 
then we adds it to a {\it root list}.
Then \doubletake{} utilizes a BFS search algorithm to verify each value of the {\it root list}. 
For each value, \doubletake{} verifies whether this value is a valid address inside a heap object.
If this value is inside an actual heap object and this object is not checked, 
\doubletake{} checks all values inside this heap objects and puts those possible values 
in the range of allocated heap objects into {\it root list} too. 
After a heap object is verified, it is marked as a reachable object. 
To mark those reachable objects, \doubletake{} utilizes the least significant bit
of block size in headers of each objects. 
Since the size of every blocks in \doubletake{} are always \texttt{power of 2},
there is no usage on least significant bit.

For most of values appearing in the {\it root list}, they are possibly starting addresses of 
heap objects. \doubletake{} also leverages the same bitmap of canaries to locate the 
starting address of this object when a value is not the starting address of an object. 

In the end, \doubletake{} traverses the whole heap to find those leaked heap objects.
Because its customized memory allocator has embedded the actual size information into object headers 
and those actual size are set to 0 when an object is freed, 
\doubletake{} identify a leaked heap object if this object is not freed and it is not reachable. 
Also, \doubletake{} clears the least significant bit of all objects 
in this check procedure for the future scanning. 
%\doubletake{} can also identify
%reachable memory that has been freed, which could potentially
%lead a use after free violation.

\subsubsection{Reporting}
\label{sec:leakreport}
\doubletake{} uses re-execution to find allocation sites for leaked heap objects. 
Re-execution proceeds as normal, with an added check in each \texttt{malloc()}.
When a memory allocation matches the actual size and address of any leaked heap object, 
\doubletake{} reports its call stack, obtaining by using \texttt{backtrace} 
functions of \texttt{glibc} library. 

\subsection{Detection of Use-after-free Errors}
\label{sec:danglingpointer}
Memory use-after-free errors occur when an application continue to use pointers, 
where those objects pointing by them have been deallocated.  
Use-after-free errors can induce unexpected behavior of programs execution 
if an de-allocated object is allocated for other purposes, 
violating correctness and security guarantee of programs. 

\subsubsection{Detection}
To detect memory use-after-free errors, 
\doubletake{} firstly delays memory reusage of all freed objects 
by putting them into a quarantine list, 
same as AddressSanitizer does ~\cite{AddressSanitizer}. 
Those objects in the quarantine list are actually returned back
to the program heap when the total size of freed objects in the quarantine list 
are larger then a pre-defined threshold or the quarantine list is full.
In order to find evidence of memory use-after-free problems, 
\doubletake{} fills the first 128 bytes of an object, which can be adjustable, 
with canaries. 

Before an object is returned back to the program heap,
all canaries inside an object are checked. 
In the end of an epoch, all heap objects inside the quarantine list are also checked. 
Same as detection of buffer overflows, 
a corrupted canary indicates a use-after-free memory error and must be reported to
user. 

\subsubsection{Reporting}
When a program has use-after-free memory errors, 
\doubletake{} re-executes this program to find out the allocation and deallocation site
of corresponding objects and those instructions actually writing them.
It is possible that multiple instructions can access an object after deallocation.  
To obtain the allocation and deallocation callsite information, 
\doubletake{} leverages the same mechansim used in
detection of memory leakage, which is discussed in Section~\ref{sec:leakreport}.
To find out those instructions writting a freed object, 
\doubletake{} installs hardware watch points on those violating addresses, which
shares the same mechanism as overflow detection (Section~\ref{sec:overflowreport}).
