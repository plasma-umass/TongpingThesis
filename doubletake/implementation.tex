\label{sec:implementation}

This section describes the implementation of \doubletake{}, organized by the phases shown in Figure~\ref{fig:overview}. 

% we should say why we have normal execution and re-execution.
\doubletake{} divides the execution of a program to different epochs at the boundary of irrevocable system calls. At each epoch, \doubletake{} takes a snapshot of the program in the beginning and lets an application run at full speed until irrevocable system calls (in the end of this epoch), while recording some of the program's operations to facilitate re-execution. Then \doubletake{} examines the state for evidence of possible memory errors. 
If there is no evidence of errors, \doubletake{} issues this system call and enters into next epoch. If there is an error detected, \doubletake{} rolls back the program to the most recent checkpoint and re-executes the program to locate the exact cause of this error. 

\subsection{Epoch Start}
{\em Snapshot}. 
At the beginning of every epoch, \doubletake{} take a snapshot of the current program state so that we can rollback to this state if there are some errors detected at the end of the current epoch. A snapshot includes the state of registers (obtained using \texttt{getcontext()}), and all writable memory, including the globals, the stack and the program heap. 
Read-only memory, such as text segments of a program and libraries, does not need to be saved in the snapshot. \doubletake{} analyzes the Linux file \texttt{/proc/self/map} 
to identify the ranges of the globals and the stack. Because of using a customized memory allocator, \doubletake{} knows the range of the heap. 

To save the snapshot, \doubletake{} saves the globals, heap, and stack before it calls \texttt{getcontext} to save its execution context. Also, \doubletake{} records file positions of all opening files. This lets programs issue \texttt{read} and \texttt{write} system calls without 
ending the current epoch. \doubletake{} uses the saved memory state, register state and file positions to ``undo'' an epoch if errors are found in the end of the current epoch. 

\subsection{Inside an Epoch}
\label{sec:inepoch}
Inside an epoch, \doubletake{} lets a program execute normally. But \doubletake{} interposes  library functions involving in system calls and heap allocations in order to set tripwires and support re-execution. 

\subsubsection{System Calls.}
\label{sec:syscall}

\input{doubletake/figure/syscalltable}

\doubletake{} allows most system calls to execute safely, maybe with efficient recording support, and ends epochs only at irrevocable system calls. Thus, \doubletake{} achieves very good performance by reducing the cost of snapshotting an epoch or checking the evidence of memory errors at the end of epochs. 
System calls are classified to five different types, described in Table~\ref{table:syscalls}.

\emph{Repeatable system calls} do not modify system state, and will always return the same result during normal execution and  re-execution. \doubletake{} does not need any special handling for this type of system calls.

\emph{Recordable system calls} may return different results if they are re-executed.
\doubletake{} records the results of this type of system calls in the normal execution and replays the saved results in re-execution phase. 

\emph{Revocable system calls} modify system state, but \doubletake{} can save the state beforehand and restore it prior to re-execution. Most of file IO fall into this category, but socket IO are treated as irrevocable system calls. For example, \texttt{write()} changes the contents and the file position of a file. \doubletake{} saves file positions of all opening files in the beginning of each epoch and recovers them before re-execution. 

\emph{Delayable system calls} irrevocably change program state, but can safely be delayed until the end of the current epoch. For example, \doubletake{} delays all calls to \texttt{munmap()} and \texttt{close()}, and executes them before exiting the current system calls. 

\emph{Irrevocable system calls} changes the program state and can not be un-done. \doubletake{} ends the current epoch before this type of system calls are allowed to proceed. A system call not belonging to previous categories can be conservatively treated as an irrevocable system call. Note that \doubletake{} uses a different meaning for ``revocable'' than that used in transactional memory systems ~\cite{Irrevocabletrans}. For example, file IO normally are treated as irrevocable system calls. When the results of re-execution can be identical to the epoch’s original
execution, it is safe for system calls to affect externally-visible state as long as the effect can be undone or reproduced. Reducing the number of irrevocable system calls helps improve the performance because of reducing unnecessary overhead of starting and ending an epoch. 

\subsubsection{Multi-threading Support}
\doubletake{} records the sequence of synchronizations and corresponding results separately for each thread. Basically, a synchronization event is recorded in two lists simultaneously, a per-thread list and a synchronization variable list.

For every mutex, \doubletake{} records the order of threads that acquire it. For each conditional variable, \doubletake{}records the order of thread wake-ups. \doubletake{} do not record the order of thread waits since this is enforced by the order of mutex. \doubletake{} does not enforce a total global order on lock acquisitions. Operations within a single thread are totally-ordered and recorded, and \doubletake{} enforces local order for each synchronization variable. In the absence of data races, it is sufficient to ensure deterministic re-execution. If it fails to reveal the error on replay, \doubletake{} can replay multiple times in order to get expected order.

Calls to \texttt{pthread\_create} are recorded with the same mechanism as recordable system calls. When a new thread starts, \doubletake{} takes a snapshot of the thread's stack and registers to enable re-execution from the beginning of the thread's execution. As with synchronization operations, \doubletake{} logs thread creation order and enforces this order during re-execution.  Threads exits are deferred until the end of the epoch. \texttt{pthread\_join} is effectively deferred as well.

\subsubsection{Multithreading Heap Allocator}
The heap allocator discussed in Section~\ref{sec:overview} can achieve deterministic replay for applications with single thread. However, it is not enough for multi-threading applications. 

To support multi-threading applications, \doubletake{} borrows a ``per-thread-heap'' idea from Hoard~\cite{Hoard}. A thread can only allocate memory from its own sub-heap, where those sub-heaps can obtain the memory from the global pre-allocated heap by acquiring a lock, which are guaranteed to be deterministic by the recording and the deterministic replay provided by \doubletake{}. When an object is freed, this object can only be returned to the subheap owned by current thread. 

\doubletake{} uses a separate heap for internal usage. Any additional memory allocations in the replay phase are also allocated from a separate heap too, such as the \texttt{backtrace} call or printing.

\subsection{Epoch End}
The epoch ends when any thread issues an irrevocable system
call. All other threads are notified with a signal. After all threads are stopped, \doubletake{} checks the program state for the evidence of memory errors. The details of checking errors for heap buffer overflows, memory leak, and memory use-after-free errors are discussed in Section~\ref{sec:applications}. 

When \doubletake{} do not find any evidence of memory errors, it issues all delayable system calls and cleans lists of recordable system calls before entering into next epoch.  
 
If \doubletake{} finds any memory error, it rolls back the execution to the most recent checkpoint and re-executes this program in order to locate the exact cause of this error. 

\subsection{Rollback}
\label{sec:rollback}
When \doubletake{} detects an error, it rolls back the program  and re-execute it in order to collect exact causes of memory errors, which can be impossible or difficult to collect in normal execution. 

\doubletake{} prepares the following things before the actual rollback. In order to locate instructions responsible for buffer overflows and use-after-free errors,
\doubletake{} installs watch points on addresses with corrupted canaries, as described in Section~\ref{sec:applications}.
Since it is impossible to access hardware debug registers in the user space, \doubletake{} forks a child process to install watch points for the current process.  
In order to locate the site of allocation and de-allocation for objects with use-after-free errors and leak, \doubletake{} puts  all erroneous heap objects into a hash map, which are checked for every memory allocation and deallocation during re-execution. \doubletake{} also recovers file positions of all opening files.

After that, \doubletake{} rolls back the state of program to the most recent snapshot before re-execution. Before the program stack can be restored, \doubletake{} must switch to a temporary stack in order to avoid the overwriting of current stack. Then \doubletake{} restores the state of all writable memory from the snapshot. After that, \doubletake{} restores register state with the \texttt{setcontext()} function and re-execution proceeds automatically.
In the multithreading case, the thread that stop other threads notifies other threads to restore their own registers in a proper time. 

\subsection{Re-Execution}
In re-execution phase, \doubletake{} handles system calls specifically depending on the type of system calls. For recordable system calls, \doubletake{} replays the saved results from the log collected during normal execution.
For deferred system calls, \doubletake{} converts them to no-ops while the program is re-executing. \doubletake{}  issues other system calls normally.

In re-execution phase, \doubletake{} are collecting information of memory errors, which are discussed in Section~\ref{sec:applications}. The call site information is always acquired by calling \texttt{backtrace} function, where all functions called by this should not involve in any replaying or the usage of the program heap. To collect the allocation site and deallocation site, \doubletake{} interposes memory allocation and deallocation. To detect the exact location causing buffer overflows and use-after-free errors, \doubletake{} handles the traps caused by accesses on watch points, setting up before the actual rollback (Section ~\ref{sec:rollback}). 

\subsubsection{Trap Handler}
Inside the trap handler, \doubletake{} firstly determines which watchpoint causes the current trap if there are multiple watch points in total. Normally, this information can be got by checking the debug status register. However, it is difficult to do this since this register can only be accessed by using \texttt{ptrace} function, which involves in more than two processes. \doubletake{} updates the values of different watch points. Thus, it can precisely determine which watchpoint is triggered by checking the changes of those watchpoints. After this, \doubletake{} also updates the values of the current watch point. 

After that, \doubletake{} checks whether the faulty instruction is issued by \doubletake{} library or not. \doubletake{} sets canaries in order to detect buffer overflows and use-after-free errors, thus there is no need to further handle this trap if it is caused by \doubletake{} library itself. 

In the end, \doubletake{}  prints the call site of buffer overflows or use-after-free errors and their memory allocation (and deallocation).  

\subsubsection{Synchronization Replay}
\doubletake{} enforces the recorded order of synchronization operations during re-execution by using the {\it semaphore replay} mechanism~\cite{TERN}. \doubletake{} assigns a binary semaphore for each thread, initialized as unavailable. Lock acquisitions and conditional wakeups are treated similarly here.

Before the rollback, \doubletake{} checks the first synchronization event of every synchronization variable list. If an synchronization event is also the first event of a thread, the semaphore of this thread is incremented immediately. Otherwise, this thread are granted with a pending increment, incremented only after all previous synchronizations of this thread has been  handled.

In the replay, a lock acquiring is turned into a semaphore wait. During a lock release, \doubletake{} actually increments  the semaphore of next thread in the same synchronization variable list, no matter whether the next thread is the same thread or not. In order to support multiple locks for a critical section, a lock acquisition also increments pending increments if possible.


