\label{sec:multithreading}

This section describes the mulithreading support of \doubletake{}.
A thread is a basic execution unit from the point of view of underlying operating system. 
The order of an execution, greatly affecting memory usage, 
is highly depending on timing, synchronization order and internal scheduling algorithm.   
Thus, it is much more difficult to achieve the target of repeatable memory 
usage for multithreading programs, which is crucial to 
precisely detect buffer overflows or other memory errors.
This section first discusses how to handle epochs in multithreading programs.
After this, it describes the design of heap allocator, suitable for repeatable memory usage.
It then discusses how to handle thread creation and exits specially. 
In the end, it describes how to guarantee deterministic synchronization in the re-execution phase,
which is also crucial for repeatable memory usage.


\subsection{Overview}
\label{sec:mtoverview}

As described in Section~\ref{sec:overview}, \doubletake{} uses irrevocable system calls as 
boundaries for epochs for multithreading programs. 
To simplify description in the following sections, a thread encountering an irrevocalbe system call is 
called as the ``Triggering-Thread''. 

When encountering an irrevocable system call, this Triggering-Thread 
has to stop all existing threads so that all other threads are in a quiecent state, which
has been described in Section~\ref{sec:stopepoch}.
Then it performs memory checkings on the heap as described in Section~\ref{sec:epochend}. 
If there is no buffer overflow, it can perform this irrevocable system call and 
start a new epoch after this system call. 
Before waking up other threads, the Triggering-Thread takes a snapshot for the shared memory at first,
including the heap and globals. 
After a thread is waken up, it only needs to take a snapshot on its own state, 
including its stack and its hardware registers. 

If there are buffer overflows, the Triggering-Thread sets up the shared memory 
for all threads at first.
It recovers the heap and globals by copying from the saved snapshot. 
Then it can wake up other threads. 
However, if the Triggering-Thread is spawed newly in current epoch, 
it has to wait for its parent to start its execution. 

\subsubsection{Epoch}
\label{sec:stopepoch}

It is the duty of a Triggering-Thread to close an epoch.
Whenever this thread meets an irrevocable system call, it has to stop other threads.
\doubletake{} utilizes the ``signal'' mechanism to stop other threads asynchronously.
It signals other threads using SIGUSR2 signal when a thread is in a safe state. 
A thread is considered to be in a unsafe state before this thread finishs a snapshot for itself,
discussed in Section~\ref{sec:threadcreation}.
After sending out all signals, this thread is waiting on a internal conditional variable.
The Triggering-Thread only starts checking buffer overflows after all threads are in quiescence.

However, this SIGUSR2 signal can also be used by user programs. 
In order to differentiate this, \doubletake{} specifically marks on 
a shared flag before signalling so that signal handler can check this in the beginning. 
If this flag is marked, this signal is issued by \doubletake{}. Otherwise, it is issued by
a user program and we can call user registered program instead. 

When other threads receive the signal from the Triggering-Thread, 
they are waiting on an internal conditional variable for instructions from the Triggering-Thread:
it can move forward to next epoch or rollback.
It is worthy noting that inside signal handler we have to utilize a different lock that has not been
used by other places. Otherwise, it is easy to cause deadlock. 

\subsubsection{Customized Heap Allocator}
\label{sec:mtheap}
In order to achive the target of repeatable memory usage, the heap allocator must be designed 
carefully. \doubletake{} first borrows a ``per-thread-heap'' idea from Hoard~\cite{Hoard}. 
\doubletake{} keeps a 1-to-1 mapping between threads and sub-heaps of customized memory allocator. 
The total number of threads and sub-heaps are pre-defined. 
A thread can only allocate memory from its own sub-heap, 
where those sub-heaps can get the memory from a pre-allocated heap 
by allocating a huge block of memory each time. 
After a sub-heap gets a block of memory, its corresponding thread always owns all objects
of this block, called as ``owner'' of this block.
This surely can cause memory blowup problem resolved by Hoard. However, it is 
not the focus of \doubletake{}.   

The ``per-thread-heap'' idea is not enough to guarantee the repeatable memory usage. 
\doubletake{} imposes several additional rules besides this.
Firstly, when \doubletake{} acquires a block of memory from the global pre-allocated heap, it must 
acquire a lock at first, which is guaranteed to be deterministic according to mechanisms 
discussed in Section~\ref{sec:sync}.
This guarantee that every new blocks of each sub-heap is repeatable for re-execution. 
Secondly, when there is a memory deallocation, this freed object can only be returned back  
to its original owner in a safe state. 
If this memory deallocation is issued by the same thread as the owner, then this freed object
can be putted into the owner's free list and be utilized immediately. 
If this memory deallocation is issued by a different thread with the owner, 
which indicates a cross-thread communication,  
then this memory
deallocation are cached into a global list, which only issued in the end of this epoch after 
all threads has been stopped. 
By doing this, we can guarantee all memory usage inside an epoch is repeatable in re-execution phase.
 
\subsection{Thread Creation and Exit}
\label{sec:threadcreation}
Tracking creations and exits of threads is very important because of the following reasons.
First, \doubletake{} has to take snapshots for different threads in the beginning. 
Second, terminination of a thread invokes \texttt{munmap} system calls directly by \pthreads{}, which
can not be intercepted.  
Third, thread creation is considered as a synchronization and has to be recorded. 
Thus, \doubletake{} intercepts \texttt{pthread\_create} calls and changes its start routine 
to a customized function. 
In this customized function, \doubletake{} can record thread creation, take a snapshot and delay thread exit to the end of current epoch. 

\subsubsection{Normal Execution}

\doubletake{} makes \texttt{pthread\_create} to call its customized function as the start routine. 
In this start routine, \doubletake{} first puts this new thread into a global map, which maintains
status of all threads. 
Then it takes a snapshot for this new thread, including stack and hardware registers. 
After this, \doubletake{} can invoke the original routine to actually perform user-defined 
thread function. 

After this user-defined thread function finishes, the control flow returns back to \doubletake{}. 
Basically, \doubletake{} should check whether this thread's parent is joining on this thread or not. 
If this thread's parent is already waiting for its termination, it simply marks the status of 
this thread to be joined and wakes up the joining thread. 
If not, this thread can wait on a thread-private conditional variables. 
\doubletake{} delays a thread exit to the end of current epoch.

\subsubsection{Re-execution}
A thread is waiting for its turn to run if a thread is created in current epoch.    

\subsection{Thread Synchronizations}

\label{sec:sync}

Different order of thread synchronizations can lead to totally different memory uage. 
In order to guarantee deterministic replay of thread synchronizations, previous work
actually forces threads to do synchronizations in a global order and
recordes both lock and unlock operations ~\cite{TERN, PRES}. 
However, forcing a global order of synchronizations can greatly 
reduce parallelism and introduce significant performance overhead.
Also, it is unnecessary to record unlock order too.

Unlike previous approaches, \doubletake{} only records local orders of synchronizations.
Synchronizations on two different synchronization variables can be performaed
in parallel. From a thread's point of view, if a program do not have a race and
all synchronizations of a thread are repeated deterministically, then
\doubletake{} can guarantee memory usage of this thread, which also guarantees to 
repeat the same buffer overflows in re-execution phase. 
If a program do have a race, forcing a global order of synchronizations in the production 
run also can not completely avoid races. This also implies that a global order 
can not always guarantee determinstic memory uage.   
\doubletake{} prefers performance for racey programs, while relying on multiple re-executions 
to repeat buffer overflows if a program does have a race problem. 

\doubletake{} records the order of \texttt{pthread\_mutex\_lock}, conditional wakenup and
different signalling functions.
Signalling functions actually calls system calls, which are handled by the procedure discussed 
in Section~\ref{sec:inepoch}.
 Conditional wakenup is actually related to \texttt{pthread\_cond\_wait},
which actually includes conditional wait and conditional wakenup phases. 
Conditional wait atomically releases mutex and waits on corresponding conditional variable, while conditional wakenup actually locks corresponding mutex before returning. 
Thus, we can turn \texttt{pthread\_cond\_wait} to two operations, \texttt{pthread\_mutex\_unlock} and
\texttt{pthread\_mutex\_lock} correspondingly. So we only record the order of conditional wakenup in
production runs. 
 
\doubletake{} also provides an option to record the order of passing a specified barrier, which it is 
not necessary to do this by default.
It is noted that \doubletake{} do not record the order of unlock operations
and conditional signal operations.
It is totally unnecessary to record unlock operations since recording the order of actual 
lock aquiring operations is enough to guarantee a deterministic replay of critical sections.  
Conditional signal and broadcast operations are skiped for the same reason. 

%why two different synchronization is not important?
How to replay this?
How to handle nested locks? 
The replaying is considerred to be two steps: we first advance thread's entry when we met a .

Maybe pseudo code for this.
 
\subsubsection{Normal Execution}
In production runs, \doubletake{} intercepts all synchronizations and 
records orders of synchronizations, such as lock, conditional waken up and signals, 
based on different synchronization variables. 
It maintains a list for each synchronization variable and records synchronization
events on its corresponding list. 
In order to quickly locate its list when a synchronization is intercepted, \doubletake{}
utilizes original synchronization variables to store addresses of list and actual synchronization
variable. 

For a synchronization event like lock, \doubletake{} records the following information:
which thread issues this synchronization event; what is the result of this synchronization.
A naive implementation is to allocate memory from internal memory allocator every time. 
However, for some applications having significant amount of synchronizations,
memory allocations to record synchronization events contributes much performance overhead. 
For example, \texttt{fluidanimate} runs several times slower because of huge amounts
of synchronizations inside. \doubletake{} uses a pre-allocated list for those recordings. 
More specifically, each thread has a pre-allocated list in the beginning of an epoch. 
When a synchronization occurs on this thread, it can get an entry from this thread and 
record a synchronization event on this entry.
Since \doubletake{} always gets an entry from current thread issuing a synchronization,
there is no need to utilize a lock, which also helps reducing overhead. 
By doing this, \doubletake{} greatly reduce performance overhead of logging synchronization
events. For example, performance overhead of \texttt{fludidanimate} are reduced to around 40\%.

 
\subsubsection{Re-execution}
As described above, for a synchronization event in a thread,
\doubletake{} allocates an entry from current thread to record this synchronization event 
and inserted it into synchronization variable's corresponding list. 
This implies that a synchronization event belongs to two lists, 
a list for all synchronizations of this synchronization variable (SyncVariableList) and 
a list for all synchronizations in a thread (ThreadSyncList). 

\begin{figure}[!ht]
{\centering
\subfigure{\lstinputlisting[numbers=none,frame=none,boxpos=t]{figure/lockunlock.pseudocode}}
\caption{Lock and unlock of reproduction runs. \label{fig:lockunlock}}
}
\end{figure}

Reproducing synchronizations involves in manipulating these two lists using 
{\it sempaphore replay}, similar to TERN ~\cite{TERN}.
We listed the pseudocode of ``lock'' of reproduction runs in Figure~\ref{fig:lockunlock}.
\doubletake{} assigns a semaphore for each thread and controls the order 
of synchronizations based on semaphores: in lock acquisitions, 
a thread waits on its semaphore and advances ThreadSyncList after this semaphore; 
In lock releases, a thread increments the semaphore of next thread on the same 
synchronization variable.
However, \doubletake{} only records local synchronization order, instead of global order,
synchronizations replaying of \doubletake{} is much more subtle.
In order to handle those unsuccessful lock acquisitions, \doubletake{} only waits for a semaphore 
if this lock is successfully acquired in the production run. 
Also, to support nesting locks, in lock acquisitions after {\it advanceThreadSyncList()}, 
\doubletake{} signals current thread if next event of this thread is already 
in its pending list, which means that this thread should have its turn.
For lock releases, \doubletake{} adds next event of SyncVariableList to corresponding thread's 
pending list if the event is not the first event of corresponding thread instead of incrementing
its semaphore directly. 
Since performance of reproduction runs is not the main focus, only occurring for those programs 
having buffer overflows, \doubletake{} are using the same lock for all lists' manipulations to
avoid races.  


