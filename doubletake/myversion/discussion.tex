\subsection{Limitations of Applications}
\doubletake{} implements three detection tools for heap buffer overflows, use-after-free errors, and memory leaks.  They are utilized to demonstrate \doubletake{}'s effectiveness, instead of providing a complete solution for these memory errors. 

The heap overflow detection tool cannot detect those non-continuous buffer overflows. If a buffer overflow does not corrupt those canaries, it cannot be detected by \doubletake{}. Memory leak detection can also miss some memory leaks. If a global variable is still pointing to a memory object, this object is not considered to be leaked even if this program do not access it at all. For performance reason, \doubletake{} only fills canaries to a freed object up to 128 bytes. For a large object, if a use-after-free error writing to a word with offset larger than 128 bytes, this use-after-free error can not be detected by \doubletake{}. 

Also, \doubletake{} can only detect those memory errors with evidence. It is impossible for \doubletake{} to detect those errors caused by out-of-bounds reads, read after free errors. 
 
\subsection{Future Work} 
One future work is to extending \doubletake{} for more usage cases. For example, based on reachability information, it is easy for \doubletake{} to find some dangling pointers, where an object has been freed but global variables still has a pointer pointing to them.

Another future work is to design a new memory allocator for detecting use-after-free errors more efficiently. Large heap objects can be allocated from a different heap, thus there is no need to fill canaries on those freed objects. We can utilize  memory protection mechanism to detect those uses on large objects.

We also plan to incorporate the \doubletake{} framework with \texttt{gdb} debugger in order to fully utilize the lightweight recording and deterministic replay mechanism of \doubletake{}. Some software bugs, e.g. relating to timing, may disappear when a program is restarted. Thus, based on \doubletake{}'s deterministic replay,  \texttt{gdb} can always repeat the occurrence of software bugs. It is also possible to provide an interactive debugging system without actually using \texttt{gdb}. 
