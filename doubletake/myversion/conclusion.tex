
This paper presents \doubletake{}, a super lightweight framework for dynamic analysis. \doubletake{} divides the execution of a program into different epochs, at irrevocable system calls. Inside an epoch, \doubletake{} allows a program to run at full speed, with very minimum recording overhead. 
At the end of an epoch, \doubletake{} checks the program state for possible errors by verifying memory error evidence. If the evidence do not naturally occur in normal execution, e.g., buffer overflows, \doubletake{} installs tripwires to ensure later detection. If an error is detected, \doubletake{} rolls back the execution to a previous checkpoint, taken in the beginning of the current epoch. During re-execution, \doubletake{} can use more expensive instrumentation in order to obtain details about an error. Because \doubletake{} ends epochs rarely, the performance overhead of checkpointing the state and checking memory overhead is amortized through long uninterrupted period, thus achieving very low performance overhead. We have implemented three tools based on \doubletake{}'s framework, including the detection of heap buffer overflows, memory leaks, and memory use-after-free errors. We also evaluate the performance overhead of \doubletake{}. For detecting heap buffer overflows and memory leaks, \doubletake{} only introduces less than 4\% performance overhead, making it the fastest tools up to date. By detecting all three errors together, \doubletake{} only introduces 9\% performance overhead. \doubletake{} is ready to be used in real deployment because of its efficiency. 

