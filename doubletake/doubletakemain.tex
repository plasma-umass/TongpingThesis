
Dynamic analysis tools are widely used to find bugs in applications. They are popular among programmers because of their precision---for many analyses, they report no false positives---and can pinpoint the exact location of errors, down to the individual line of code.

Perhaps the most prominent and widely used dynamic analysis tool for C/C++ applications is Valgrind~\cite{Valgrind}. Valgrind's most popular use case is to check memory errors, including buffer overflows, memory use-after-free errors, and memory leaks.

However, while these dynamic analysis tools are very useful, they are often expensive. Using Valgrind normally slows down applications around $10\times - 100\times$. The most recent work, Google's AddressSanitizer, still imposes about 30\% performance overhead on the detection of buffer overflows and memory use-after-free errors ~\cite{AddressSanitizer}. The significant performance overhead limits the usage of these tools only in the debugging phase. However, it is impossible to discover all bugs in the debugging phase, because a lot of bugs happens either with some specific inputs or at a specific running environment. For example, most discovered buffer overflow errors only occurs when they are feed with a specific input. Thus, it is necessary and helpful to have a detection tool with extremely low overhead so that it can be used in real deployment environment.  

This paper presents a novel dynamic analysis tool with extremely lightweight overhead, which can detect an important class of memory errors that sharing the monotonicity property: when an error occurs, the evidence of its occurrence either remains or grows in the memory so that it can be discovered at a later time. When an evidence is not naturally in the program, it is often possible to plant evidence in order to help a later discovery. For example, existing tools place a canary (a random value) around allocated heap objects, thus they can detect buffer overflows by checking the corruptions of those canaries. 

% What is insight
We present a evidence-triggered dynamic analysis, \doubletake{}, with the following insight: by combining checkpointing with evidence planting, it is possible to let applications run at full speed in the common case (no errors). If we discover the evidence of an error, we can roll back and re-execute the program with instrumentation in order to precisely locate the error. 


\doubletake{} lets applications run at full speed until irrevocable system calls. Then \doubletake{} examines the memory for the evidence of possible memory errors. If it can not find any evidence of errors (common case), \doubletake{} performs checkpointing after irrevocable system calls, by saving the contents of the stack, globals, the heap and registers.  If there is an error detected, \doubletake{} rolls back the program to the most recent checkpoint and re-execute the program. During re-execution of the program, \doubletake{} are collecting necessary information helps diagnose the error. Because re-execution only happens when the program has errors (not the common case), we can employ some heavier techniques in the program re-execution. For example, in order to locate where false sharing occurs, \doubletake{} employs the watchpoint mechanism to keep track of all memory accesses at the corrupted address. Because the overhead of checkpointing and memory checking is amortized over a long execution, \doubletake{} achieves much less overhead than existing tools, where most of them has to intercept every memory access in order to report a memory error timely.

\doubletake{} is a drop-in library that can be linked to the application directly, or be preloaded before execution by setting the LD\_PRELOAD environment variable. 
Because of that, there is no need to re-compile the code, as convenient as using Valgrind.     

We have built three different applications using \doubletake{}: buffer overflow detection, use-after-free detection and memory leak detection. All of these applications run without any false positive, precisely pinpoint the error location, and operate with extremely low overhead. For example, \doubletake{} only runs with just 3\% performance overhead for buffer overflow detection and memory leak detection together, which makes it the fastest detection tool up to date. 

\section{Overview}
\label{sec:overview}

\begin{figure}[!t]
\begin{center}
\includegraphics[width=3.3in]{doubletake/figure/overview}
\end{center}
\caption{
Overview of \doubletake{}: execution is divided into epochs at the boundary of irrevocable system calls. 
\label{fig:overview}}
\end{figure}

\doubletake{} aims to reduce the performance overhead of dynamic analyses for memory errors sharing the monotonicity: the evidence of an error is persistent and can be detected after-the-fact. As described in Figure~\ref{fig:overview}, the execution of a program is divided into epochs at irrevocable system calls, discussed in Section~\ref{sec:normal_execution}. Inside each epoch, \doubletake{} allows a program to run at full speed, with the support of checkpointing and very minimum recording overhead. \doubletake{} only checks for evidence of an error when an epoch ends. If an error is detected, \doubletake{} rolls back to the most recent checkpoint and re-execute the program to pinpoint the error.  During the re-execution phase, \doubletake{} can use higher-overhead mechanisms to pinpoint the exact cause of the error. 

Based on this framework, we have implemented three detection tools for heap buffer overflows, use-after-free errors, and memory leaks, which are described in detail in Section~\ref{sec:applications}. 

\doubletake{} employs the following core mechanisms:

\paragraph{Efficient Recording.}
At the beginning of every epoch, \doubletake{} saves a snapshot of program registers, and all writable memory. The epoch ends when the program attempts to issue an irrevocable system call, which is described in Section~\ref{} \CC{here}. \doubletake{} also records the order of thread synchronization operations to support re-execution of parallel programs. \doubletake{} records minimal system state at the beginning of each epoch (like file offsets), which allows system calls that modify this state to be undone if re-execution is required. As a result, most programs require very few epochs and program state checks. We describe the details of each application's state checks in Section~\ref{sec:applications}.

\paragraph{Precise Replay.}
During re-execution, \doubletake{} ensures that all observable system state, system call results, and memory allocations will be the same from the original run. System calls that cannot be replayed, called irrevocable system calls, must be issued at the end of an epoch after detection tools have verified that no errors have occurred. Actually, those system calls consists of the boundary of epochs, which are discussed in Section~\ref{sec:normal_execution}. In practice, most system calls can easily be replayed by handling appropriately. This ensure that most programs run very few integrity checks, and overhead remains low when errors are not detected.

\paragraph{Custom Heap Allocator.}

\doubletake{} replaces the default heap allocator with a BiBOP-style memory allocator, built on HeapLayers~\cite{heaplayers}. \DoubleTake{} pre-allocates a fixed size of memory from its underlying operating system using \texttt{mmap} system calls and satisfies memory allocations from this block by interposing all memory allocation and deallocations. Using the custom memory allocators avoids a big number of sbrk() or mmap() system calls caused by the default memory allocator. In the heap, all heap objects have the block size of {\it power of $2$}, using an object header to mark its status and size information. There is no split and merge operation on heap objects. If the size of an allocation is less than {\it power of 2}, \DoubleTake{} allocates an object with the size of next {\it power of 2}. 






\section{Applications}
\label{sec:applications}

\subsection{Shared Mechanisms}
Before the description of different applications, we listed some shared mechanisms that are used by one or multiple applications. 

\subsubsection{Canary}
\label{sec:canary}
Canary was first proposed by StackGuard~\cite{StackGuard} to find stack smashing problems by placing a canary word before the return address on stack. Those attempts to overwrite the return address should corrupt the canary word at first. Canaries are borrowed to detect buffer overflow ~\cite{overflow:purify}. They are initialized to a special value in the beginning so that modifications of those values indicates the problems. Detection tools can place canary values anywhere in heap memory. \doubletake{} introduces a bitmap to track the locations of canary values, and can check for canaries that have been overwritten. Buffer overflow detection places canary values between heap objects to detect out-of-bounds writes, and use-after-free detection fills freed objects with canaries to detect writes through dangling pointers.

\subsubsection{Watchpoints}
\label{sec:watchpoints}
Watch point mechanism relies on hardware debug registers to monitor memory accesses on specific addresses. Previous work has used this mechanism for their specific targets ~\cite{fastboundschecking, Kivati}. The main target of debug registers is to support software debuggers, e.g. gdb. A small number of watchpoints are available on modern architectures (four on x86). Each watchpoint can be configured to pause program execution when a specific byte or word of memory is accessed. \doubletake{} allows detection tools to set hardware watchpoints before re-execution. Heap Overflow detection tool can use watchpoints to find the instruction(s) responsible for overwriting a canary value.

\subsection{Applications}
\label{sec:applications}

\doubletake{} implements three important applications based on its lightweight dynamic analysis framework. Those applications are implemented in a best-effort way to show the efficiency of our framework. They are not targeted for a complete and novel solution, thus most of mechanisms are borrowed from previous work. 

\begin{figure}[!t]
\begin{center}
\includegraphics[width=3.3in]{doubletake/figure/buffer-overflow-diagram}
\end{center}
\caption{Heap organization used to provide evidence of buffer overflow errors. Object headers and unrequested space within allocated objects are filled with canaries; a corrupted canary indicates an overflow occurred.
\label{fig:buffer-overflow}}
\end{figure}

\begin{figure}[!t]
\begin{center}
\includegraphics[width=3.3in]{doubletake/figure/dangling-pointer-diagram}
\end{center}
\caption{Evidence-based detection of dangling pointer (use-after-free) errors. Freed objects are deferred in a quarantine in FIFO order and filled with canaries. A corrupted canary indicates that a write was performed after an object was freed.
\label{fig:dangling-pointer}}
\end{figure}


\subsubsection{Detection of Heap Buffer Overflows}
\label{sec:overflow}
Buffer overflows occurs when a program writes outside of the range of an allocated object. Buffer overflows can greatly affect the reliability and security of a program. 

\paragraph{Detection.}
To detect heap buffer overflows, \DoubleTake{} puts canaries before and after actual heap objects, which is described in Figure~\ref{fig:buffer-overflow}. This method is borrowed from previous work~\cite{overflow:lbc, AddressSanitizer}.
All heap objects of \doubletake{} are managed by power of two size, adding two words for canaries (16 bytes) for each heap object. For an allocation size not power of two size (a non-aligned object), \doubletake{} rounds it up to the next power of two size class, putting byte-based canaries and word-based canaries immediately after this object. This approach lets \doubletake{} catch overflows as small as one byte. 

Detection happens in the following scenario. At memory deallocation, \doubletake{} checks buffer overflows only for non-aligned objects, with size different with the power of two class. At the end of each epoch, \doubletake{} verifies integrities of all canaries by traversing the bitmap, which marks the placement of canaries and discussed in Section~\ref{sec:canaries}.  Corrupted canaries indicates heap buffer overflows. 

\paragraph{Re-execution.}
\label{sec:overflowreport}
When a program is found to have heap overflows, \doubletake{} rolls back the execution to the most recent checkpoint and re-executes this program.
To precisely identifying instructions responsible for a buffer overflow, \doubletake{} installs a watch point at the address of a corrupted canary before re-execution. When the program is re-executed, any instruction that writes to this address will trigger a debug trap (resulting in a SIGTRAP signal). \doubletake{} then reports the exact call site of trapped instructions, acquired by calling \texttt{backtrace} function. To further help programmer, \doubletake{} can also report the allocation site of an overflowed heap object. 
  
\subsubsection{Detection of Memory Leak}
\label{sec:leak}
Heap memory is leaked when it becomes inaccessible without being freed. Memory leak can greatly reduce the performance and availability of programs, which makes it one of common classes of reported bugs.

\paragraph{Detection.}
\doubletake{} detects memory leak using the same marking mechanism as conservative garbage collection ~\cite{Wilson:1992:UGC:645648.664824}. \doubletake{} marks the reachability of all heap objects: whether a heap object is reachable from the globals, the stack, and registers. Any unreachable object that has not been freed must be leaked. 

In order to compute the reachability from globals, stack, and registers, \doubletake{} starts to put all possible pointers, any eight-byte aligned value falling into the range of the heap, into a work queue. Then \doubletake{} checks all values in the queue using a breadth-first search algorithm: for each value, \doubletake{} verifies whether this value is a valid address inside a heap object; If this valid object is still non-reachable (not checked before), \doubletake{} marks it as reachable and also put all possible pointers inside this object into the work queue. After this step, this value is removed from the work queue. \doubletake{} stops when there is no value in the work queue any more. In this time, all heap objects which is reachable should have been marked. 

After the marking phase, \doubletake{} traverses the whole heap to find leaked objects, those un-reachable non-freed objects. To simplify the marking and checking phase, the status of an object, whether it is freed or reachable are marked in the object header. During this traverse, \doubletake{} recovers the status of every object to un-reachable. \doubletake{} also adds all leaked objects into a global hash table, where each leaked object also keeps information of its starting address and size. 

\paragraph{Re-execution.}
\label{sec:leakreport}
\doubletake{} uses re-execution to find allocation sites for all leaked heap objects. Re-execution proceeds as normal, with an added check in each \texttt{malloc()} function call. When a memory allocation matches the actual size and address of any leaked heap object, 
\doubletake{} reports its call stack, obtaining by using \texttt{backtrace} 
functions of \texttt{glibc} library. 

\subsubsection{Detection of Use-after-free Errors}
\label{sec:danglingpointer}
Memory use-after-free errors occur when an application continues to use a pointer after its corresponding object has been deallocated.
Writing to a freed memory can overwrite the contents of lived objects, leading to unexpected behavior.  

\paragraph{Detection.}
To detect memory use-after-free errors, 
\doubletake{} firstly delays memory re-usage of all freed objects 
by putting them into a quarantine list, same as AddressSanitizer does ~\cite{AddressSanitizer}. 
Those objects in the quarantine list are actually returned back
to the program heap when the total size of freed objects in the quarantine list are larger then a pre-defined threshold or the quarantine list is full.

In order to find evidence of memory use-after-free problems, 
\doubletake{} fills the first 128 bytes of an object, which can be adjustable, with canaries. 

Those canaries are checked before an object is returned back to the program heap or in the end of an epoch. Same as detection of buffer overflows, 
a corrupted canary indicates a use-after-free memory error and must be reported to user. 

\paragraph{Re-execution}
When a program has use-after-free memory errors, \doubletake{} re-executes this program to find out the allocation and deallocation site of corresponding objects and those instructions actually writing them.
%It is possible that multiple instructions can access an object after deallocation.  

To obtain an object's call site of allocation and deallocation, 
\doubletake{} checks starting address of each object during every memory allocation and deallocation. If an object has the same address as those use-after-free objects, its corresponding allocation/deallocation site are saved. To find out those instructions writting a freed object, 
\doubletake{} installs hardware watch points on those violating addresses, which shares the same mechanism as overflow detection.
\doubletake{} reports an use-after-free error, with its allocation site and deallocation site, in order to help programmers locate a memory use-after-free error. 

\section{Implementation} 
\label{sec:implementation}

This section describes the implementation of \doubletake{}, organized by the phases shown in Figure~\ref{fig:overview}. 

% we should say why we have normal execution and re-execution.
\doubletake{} divides the execution of a program to different epochs at the boundary of irrevocable system calls. At each epoch, \doubletake{} takes a snapshot of the program in the beginning and lets an application run at full speed until irrevocable system calls (in the end of this epoch), while recording some of the program's operations to facilitate re-execution. Then \doubletake{} examines the state for evidence of possible memory errors. 
If there is no evidence of errors, \doubletake{} issues this system call and enters into next epoch. If there is an error detected, \doubletake{} rolls back the program to the most recent checkpoint and re-executes the program to locate the exact cause of this error. 

\subsection{Epoch Start}
{\em Snapshot}. 
At the beginning of every epoch, \doubletake{} take a snapshot of the current program state so that we can rollback to this state if there are some errors detected at the end of the current epoch. A snapshot includes the state of registers (obtained using \texttt{getcontext()}), and all writable memory, including the globals, the stack and the program heap. 
Read-only memory, such as text segments of a program and libraries, does not need to be saved in the snapshot. \doubletake{} analyzes the Linux file \texttt{/proc/self/map} 
to identify the ranges of the globals and the stack. Because of using a customized memory allocator, \doubletake{} knows the range of the heap. 

To save the snapshot, \doubletake{} saves the globals, heap, and stack before it calls \texttt{getcontext} to save its execution context. Also, \doubletake{} records file positions of all opening files. This lets programs issue \texttt{read} and \texttt{write} system calls without 
ending the current epoch. \doubletake{} uses the saved memory state, register state and file positions to ``undo'' an epoch if errors are found in the end of the current epoch. 

\subsection{Inside an Epoch}
\label{sec:inepoch}
Inside an epoch, \doubletake{} lets a program execute normally. But \doubletake{} interposes  library functions involving in system calls and heap allocations in order to set tripwires and support re-execution. 

\subsubsection{System Calls.}
\label{sec:syscall}

\input{doubletake/figure/syscalltable}

\doubletake{} allows most system calls to execute safely, maybe with efficient recording support, and ends epochs only at irrevocable system calls. Thus, \doubletake{} achieves very good performance by reducing the cost of snapshotting an epoch or checking the evidence of memory errors at the end of epochs. 
System calls are classified to five different types, described in Table~\ref{table:syscalls}.

\emph{Repeatable system calls} do not modify system state, and will always return the same result during normal execution and  re-execution. \doubletake{} does not need any special handling for this type of system calls.

\emph{Recordable system calls} may return different results if they are re-executed.
\doubletake{} records the results of this type of system calls in the normal execution and replays the saved results in re-execution phase. 

\emph{Revocable system calls} modify system state, but \doubletake{} can save the state beforehand and restore it prior to re-execution. Most of file IO fall into this category, but socket IO are treated as irrevocable system calls. For example, \texttt{write()} changes the contents and the file position of a file. \doubletake{} saves file positions of all opening files in the beginning of each epoch and recovers them before re-execution. 

\emph{Delayable system calls} irrevocably change program state, but can safely be delayed until the end of the current epoch. For example, \doubletake{} delays all calls to \texttt{munmap()} and \texttt{close()}, and executes them before exiting the current system calls. 

\emph{Irrevocable system calls} changes the program state and can not be un-done. \doubletake{} ends the current epoch before this type of system calls are allowed to proceed. A system call not belonging to previous categories can be conservatively treated as an irrevocable system call. Note that \doubletake{} uses a different meaning for ``revocable'' than that used in transactional memory systems ~\cite{Irrevocabletrans}. For example, file IO normally are treated as irrevocable system calls. When the results of re-execution can be identical to the epoch’s original
execution, it is safe for system calls to affect externally-visible state as long as the effect can be undone or reproduced. Reducing the number of irrevocable system calls helps improve the performance because of reducing unnecessary overhead of starting and ending an epoch. 

\subsubsection{Multi-threading Support}
\doubletake{} records the sequence of synchronizations and corresponding results separately for each thread. Basically, a synchronization event is recorded in two lists simultaneously, a per-thread list and a synchronization variable list.

For every mutex, \doubletake{} records the order of threads that acquire it. For each conditional variable, \doubletake{}records the order of thread wake-ups. \doubletake{} do not record the order of thread waits since this is enforced by the order of mutex. \doubletake{} does not enforce a total global order on lock acquisitions. Operations within a single thread are totally-ordered and recorded, and \doubletake{} enforces local order for each synchronization variable. In the absence of data races, it is sufficient to ensure deterministic re-execution. If it fails to reveal the error on replay, \doubletake{} can replay multiple times in order to get expected order.

Calls to \texttt{pthread\_create} are recorded with the same mechanism as recordable system calls. When a new thread starts, \doubletake{} takes a snapshot of the thread's stack and registers to enable re-execution from the beginning of the thread's execution. As with synchronization operations, \doubletake{} logs thread creation order and enforces this order during re-execution.  Threads exits are deferred until the end of the epoch. \texttt{pthread\_join} is effectively deferred as well.

\subsubsection{Multithreading Heap Allocator}
The heap allocator discussed in Section~\ref{sec:overview} can achieve deterministic replay for applications with single thread. However, it is not enough for multi-threading applications. 

To support multi-threading applications, \doubletake{} borrows a ``per-thread-heap'' idea from Hoard~\cite{Hoard}. A thread can only allocate memory from its own sub-heap, where those sub-heaps can obtain the memory from the global pre-allocated heap by acquiring a lock, which are guaranteed to be deterministic by the recording and the deterministic replay provided by \doubletake{}. When an object is freed, this object can only be returned to the subheap owned by current thread. 

\doubletake{} uses a separate heap for internal usage. Any additional memory allocations in the replay phase are also allocated from a separate heap too, such as the \texttt{backtrace} call or printing.

\subsection{Epoch End}
The epoch ends when any thread issues an irrevocable system
call. All other threads are notified with a signal. After all threads are stopped, \doubletake{} checks the program state for the evidence of memory errors. The details of checking errors for heap buffer overflows, memory leak, and memory use-after-free errors are discussed in Section~\ref{sec:applications}. 

When \doubletake{} do not find any evidence of memory errors, it issues all delayable system calls and cleans lists of recordable system calls before entering into next epoch.  
 
If \doubletake{} finds any memory error, it rolls back the execution to the most recent checkpoint and re-executes this program in order to locate the exact cause of this error. 

\subsection{Rollback}
\label{sec:rollback}
When \doubletake{} detects an error, it rolls back the program  and re-execute it in order to collect exact causes of memory errors, which can be impossible or difficult to collect in normal execution. 

\doubletake{} prepares the following things before the actual rollback. In order to locate instructions responsible for buffer overflows and use-after-free errors,
\doubletake{} installs watch points on addresses with corrupted canaries, as described in Section~\ref{sec:applications}.
Since it is impossible to access hardware debug registers in the user space, \doubletake{} forks a child process to install watch points for the current process.  
In order to locate the site of allocation and de-allocation for objects with use-after-free errors and leak, \doubletake{} puts  all erroneous heap objects into a hash map, which are checked for every memory allocation and deallocation during re-execution. \doubletake{} also recovers file positions of all opening files.

After that, \doubletake{} rolls back the state of program to the most recent snapshot before re-execution. Before the program stack can be restored, \doubletake{} must switch to a temporary stack in order to avoid the overwriting of current stack. Then \doubletake{} restores the state of all writable memory from the snapshot. After that, \doubletake{} restores register state with the \texttt{setcontext()} function and re-execution proceeds automatically.
In the multithreading case, the thread that stop other threads notifies other threads to restore their own registers in a proper time. 

\subsection{Re-Execution}
In re-execution phase, \doubletake{} handles system calls specifically depending on the type of system calls. For recordable system calls, \doubletake{} replays the saved results from the log collected during normal execution.
For deferred system calls, \doubletake{} converts them to no-ops while the program is re-executing. \doubletake{}  issues other system calls normally.

In re-execution phase, \doubletake{} are collecting information of memory errors, which are discussed in Section~\ref{sec:applications}. The call site information is always acquired by calling \texttt{backtrace} function, where all functions called by this should not involve in any replaying or the usage of the program heap. To collect the allocation site and deallocation site, \doubletake{} interposes memory allocation and deallocation. To detect the exact location causing buffer overflows and use-after-free errors, \doubletake{} handles the traps caused by accesses on watch points, setting up before the actual rollback (Section ~\ref{sec:rollback}). 

\subsubsection{Trap Handler}
Inside the trap handler, \doubletake{} firstly determines which watchpoint causes the current trap if there are multiple watch points in total. Normally, this information can be got by checking the debug status register. However, it is difficult to do this since this register can only be accessed by using \texttt{ptrace} function, which involves in more than two processes. \doubletake{} updates the values of different watch points. Thus, it can precisely determine which watchpoint is triggered by checking the changes of those watchpoints. After this, \doubletake{} also updates the values of the current watch point. 

After that, \doubletake{} checks whether the faulty instruction is issued by \doubletake{} library or not. \doubletake{} sets canaries in order to detect buffer overflows and use-after-free errors, thus there is no need to further handle this trap if it is caused by \doubletake{} library itself. 

In the end, \doubletake{}  prints the call site of buffer overflows or use-after-free errors and their memory allocation (and deallocation).  

\subsubsection{Synchronization Replay}
\doubletake{} enforces the recorded order of synchronization operations during re-execution by using the {\it semaphore replay} mechanism~\cite{TERN}. \doubletake{} assigns a binary semaphore for each thread, initialized as unavailable. Lock acquisitions and conditional wakeups are treated similarly here.

Before the rollback, \doubletake{} checks the first synchronization event of every synchronization variable list. If an synchronization event is also the first event of a thread, the semaphore of this thread is incremented immediately. Otherwise, this thread are granted with a pending increment, incremented only after all previous synchronizations of this thread has been  handled.

In the replay, a lock acquiring is turned into a semaphore wait. During a lock release, \doubletake{} actually increments  the semaphore of next thread in the same synchronization variable list, no matter whether the next thread is the same thread or not. In order to support multiple locks for a critical section, a lock acquisition also increments pending increments if possible.




\section{Evaluation}
\label{sec:evaluation}

We evaluate \doubletake{} on a quiescent Intel Core 2 dual-processor system with 16GB of RAM running Linux 2.6.18-194.17.1.el5, and version 2.5 of \texttt{glibc}. Each processor is a 4-core 64-bit Intel Xeon, operating at 2.33GHz with a 4MB shared L2 cache a 32KB per-core L1 cache. All benchmarks are built as 64-bit executables using LLVM 3.2 with the clang front-end and \texttt{-O2} optimizations.

Our evaluation measure the runtime and memory overhead of \doubletake{}, and the effectiveness of the heap buffer overflow, memory leak, and use-after-free detectors.

\subsection{Performance Overhead}
\label{sec:perf}

\begin{figure*}[!ht]
	\begin{center}
		\includegraphics[width=6.5in]{doubletake/figure/perf}
	\end{center}
	\caption{This figure shows the runtime overhead of \doubletake{} (OD - Buffer Overflow Detection, LD - Leak Detection, \doubletake{} - with three detections enabled) and AddressSanitizer, normalized to each benchmark's original execution time. 
%Overhead for Valgrind is reported in Table~\ref{table:valgrind} because the results do not fit on this graph.
\label{fig:perf}}
\end{figure*}

\begin{table}[t]
	\centering
	\begin{tabular}{r|c p{0.1em} r|c}
		\textbf{Benchmark} & \textbf{Overhead} & & \textbf{Benchmark} & \textbf{Overhead} \\
		\cline{1-2} \cline{4-5}
		400.perlbench	& 20.5X	& & 458.sjeng	& 20.3X	\\
		401.bzip2		& 16.8X	& & 471.omnetpp	& 13.9X	\\
		403.gcc			& 18.7X	& & 473.astar	& 11.9X	\\
		429.mcf			& 4.5X 	& & 433.milc		& 11.0X	\\
		445.gobmk		& 28.9X	& & 444.namd		& 24.9X	\\
		456.hmmer		& 13.8X	& & 450.dealII	& 42.8X	\\
	\end{tabular}
	\caption{Valgrind's runtime overhead. \label{table:valgrind}}
\end{table}


We evaluate performance on all C and C++ SPEC CPU2006 benchmarks, 19 applications in total. We compare \doubletake{} with AddressSanitizer and Valgrind. AddressSanitizer is the previous state-of-the-art for detecting buffer overflows and use-after-free errors~\cite{AddressSanitizer}, but cannot detect memory leaks. Valgrind's Memcheck tool is widely used tool to detect buffer overflows, memory leaks, and use-after-free errors~\cite{overflow:valgrind}. 

During performance evaluation, we disable \doubletake{}'s rollback functionality to only measure the overhead of normal execution. \doubletake{} only detects memory errors of the heap, thus AddressSanitizer is run without checks on accesses to the stack and globals, and without checks on read accesses. For each benchmark, we report the average of three runs with the largest input size, except for Valgrind. We only run Valgrind once because of its slowness. 

Performance results of \doubletake{} and AddressSanitizer are shown in Figure~\ref{fig:perf}. Results for Valgrind do not fit on the graph, and are presented separately in Table~\ref{table:valgrind}. On average, \doubletake{} adds only $9\%$ overhead \emph{with all three error detectors enabled}. If \doubletake{} do not detect memory use-after-free errors, the performance overhead is under 3\% on average. AddressSanitizer slows execution by $30\%$ on average, and Valgrind has an average overhead of $19X$ on all evaluated benchmarks. Because Valgrind is running too slow, we haven't finished the evaluation on all benchmarks. Also, we kill a program if it is already running $20\times$ slower, including \texttt{400.perlbench} and \texttt{458.sjeng}. 

% difference across all different tools
For 17 out of 19 benchmarks, \doubletake{} outperforms AddressSanitizer, even with an additional memory leak detection. For 12 benchmarks, \doubletake{}'s runtime overhead is under 3\%. \doubletake{} substantially outperforms Valgrind for all benchmarks. For both \doubletake{} and AddressSanitizer,  \texttt{400.perlbench}, \texttt{403.gcc} and \texttt{447.deallIII} benchmark introduce much more performance overhead than other benchmarks. We also observe that they all introduce much more memory overhead (in terms of absolute value), according our experimental results in Table~\ref{tbl:memoryoverhead}. These significant memory overhead may reduce the ratio of cache efficiency, thus causing performance problem. 



% Difference across all different tools
From the figure, we also notice that detection of memory use-after-free adds about 6\% performance overhead averagely, although most of overhead are coming from \texttt{400.perlbench}, \texttt{403.gcc} and \texttt{464.h264ref} benchmark. As described in Section~\ref{sec:danglingpointer}, \doubletake{} has to fill every freed object up to 128 bytes with canaries and check those canaries when a freed object is released to the program heap. If an application has a big number of memory allocation and de-allocation, these operations consist of most of performance overhead. 




Table 3 shows detailed benchmark characteristics. The “Processes” column shows the number of different invocations in the input set. The number of epochs is significantly lower than the number of system calls because of \doubletake{}'s lightweight system call handling. Benchmarks with the highest overhead run a substantial number of epochs (perlbench and h264ref) and make a large number of malloc calls (gcc, omnetpp, and
xalancbmk).



\subsection{Memory Overhead}
\label{sec:memoverhead}

\begin{table}[t]
\centering
\begin{tabular}{l|c|c|c|}
\textbf{ \small Benchmark} & \textbf{\small Original} &  \textbf{\small AddressSanitizer} & \textbf{\small \doubletake{} } \\
\hline
400.perlbench & 656 &	1481 & 1977 \\
401.bzip2	& 870 &	1020 &	1003 \\
403.gcc	& 683 &	2293 &	1583 \\
429.mcf	& 1716 &	1951 &	1994 \\
445.gobmk &	28 &	137 &	58 \\
456.hmmer &	24 &	256 &	129 \\
458.sjeng & 179 & 220 &	203 \\
462.libquantum	& 66 &	144 &	131 \\
464.h264ref	& 65 &	179 &	247 \\
471.omnetpp	& 172 &	538 &	291 \\
473.astar	& 333 &	923 &	477 \\
483.xalancbmk &	428 & 1149 &	801 \\
433.milc	& 695 &	1008 &	917 \\
444.namd	& 46 &	79 &	92 \\
447.dealII	& 514 &	2496 &	1727 \\
450.soplex	& 441 &	1991 &	1654 \\
453.povray	& 3 &	133 &	50 \\
470.lbm	& 418 &	496 &	470 \\
482.sphinx3 &	45 &	181 & 98 \\
\hline
Total & 7386 & 16678 & 13906 \\
\hline
\end{tabular}
\caption{Memory Usage of \doubletake{} and AddressSanitizer(MB).\label{tbl:memoryoverhead}}
\end{table}


\begin{figure*}
\begin{center}
\includegraphics[width=6.5in]{doubletake/figure/memory}
\end{center}
\caption{
Memory overhead of \doubletake{} and AddressSanitizer.
\label{fig:memory}}
\end{figure*}

Memory overhead of \doubletake{} comes from the following aspects. Firstly, snapshot in the beginning of each epoch, by backing up the globals, the heap and the stack, contributes the most significant memory overhead of \doubletake{}. Snapshot can double the memory usage. However, the first snapshot happening in the beginning of a program normally doesn't take too much memory since there is no heap usage at all. Thus, this explains why \texttt{401.bzip2}, \texttt{429.mcf}, \texttt{458.sjeng},  \texttt{433.milc}, and \texttt{470.lbm} have less than $2\times$ memory overhead. Secondly, recording results of system calls introduce some memory overhead. Additionally, different applications may introduce different memory overhead. For the detection of heap buffer overflows and memory leakage, \doubletake{} adds canaries around each heap object and maintains a bit map to indicate canary locations. For the detection of memory usage-after-free errors, \doubletake{} delays memory re-usage by putting freed objects into a quarantine list, which also introduces constant additional memory overhead. 

We only evaluate the physical memory overhead here because \doubletake{} pre-allocates a huge block of heap, which is 4GB virtual memory and should not be counted as memory overhead. Also, we all only care about physical memory overhead when virtual memory overhead is practically infinite in 64bit machine. Proportional set size (PSS) in \texttt{/proc/self/smaps} reflects physical memory increase on the existing system by running an application. Thus, we periodically collect this data and use the sum of different memory mappings as total physical memory usage. We presents the normalized memory overhead of running different benchmarks in Figure~\ref{fig:memory}. We also list the actual memory usage of \doubletake{} and AddressSanitizer in Table~\ref{tbl:memoryoverhead}.

From Figure~\ref{fig:memory}, \doubletake{}'s memory overhead is 2.8$\times$, while AddressSanitizer's overhead is 4.8$\times$. For \texttt{453.povray} and \texttt{464.h264ref}, both AddressSanitizer and \doubletake{} has very high normalized memory overhead because the original memory usage of this benchmark is extremely low, only 3 and 24 megabytes. But for other benchmarks, both AddressSanitizer and \doubletake{} has memory overhead lesss than $5\times$. For all benchmarks except \texttt{400.perlbench} and \texttt{444.namd}, \doubletake{} has lower memory overhead. 
From Table~\ref{tbl:memoryoverhead}, AddressSanitizer totally spends about 20\% more memory than \doubletake{}. In total, \doubletake{} memory overhead is less than 2$\times$ of original memory usage. 

\subsection{Effectiveness}
\label{sec:effect}


We use \doubletake{} to find errors in both the SPEC CPU2006
benchmark suite and a suite of real applications.

\emph{Benchmarks.} During the evaluation of SPEC CPU2006 benchmarks, \doubletake{} detected a one-byte heap buffer overflow
 in \texttt{perlbench}, which can not be detected by AddressSanitizer. \doubletake{} also detected a big number of memory leaks in \texttt{perlbench} and \texttt{gcc}, which has been verified using Valrind.

\emph{Real Applications.} For buffer overflows, we have evaluated the effectiveness on  six real applications. \texttt{libHX} has been evaluated in previous work~\cite{overflow:Cruiser}. \texttt{bzip2}~\cite{bzip2overflow} and \texttt{vim} ~\cite{vimoverflow} are got from Red Hat Bugzilla. Other benchmarks are from bugbench~\cite{bugbench}. We especially turn global array buffer overflow in \texttt{bzip2}, \texttt{gzip}, and \texttt{polymorphy} to heap buffer overflows. \doubletake{} successfully caught all these found heap buffer overflows. The buffer overflows we observed in these applications are triggered by specific inputs, which are difficult to detect during development. \doubletake{}'s overhead is low enough to be enabled in deployment, which would make it possible to detect these bugs in the field.
\doubletake{} also detects memory leaks in \texttt{gcc-4.4.7} and \texttt{vim}.
