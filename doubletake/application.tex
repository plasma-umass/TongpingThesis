\label{sec:applications}

\subsection{Shared Mechanisms}
Before the description of different applications, we listed some shared mechanisms that are used by one or multiple applications. 

\subsubsection{Canary}
\label{sec:canary}
Canary was first proposed by StackGuard~\cite{StackGuard} to find stack smashing problems by placing a canary word before the return address on stack. Those attempts to overwrite the return address should corrupt the canary word at first. Canaries are borrowed to detect buffer overflow ~\cite{overflow:purify}. They are initialized to a special value in the beginning so that modifications of those values indicates the problems. Detection tools can place canary values anywhere in heap memory. \doubletake{} introduces a bitmap to track the locations of canary values, and can check for canaries that have been overwritten. Buffer overflow detection places canary values between heap objects to detect out-of-bounds writes, and use-after-free detection fills freed objects with canaries to detect writes through dangling pointers.

\subsubsection{Watchpoints}
\label{sec:watchpoints}
Watch point mechanism relies on hardware debug registers to monitor memory accesses on specific addresses. Previous work has used this mechanism for their specific targets ~\cite{fastboundschecking, Kivati}. The main target of debug registers is to support software debuggers, e.g. gdb. A small number of watchpoints are available on modern architectures (four on x86). Each watchpoint can be configured to pause program execution when a specific byte or word of memory is accessed. \doubletake{} allows detection tools to set hardware watchpoints before re-execution. Heap Overflow detection tool can use watchpoints to find the instruction(s) responsible for overwriting a canary value.

\subsection{Applications}
\label{sec:applications}

\doubletake{} implements three important applications based on its lightweight dynamic analysis framework. Those applications are implemented in a best-effort way to show the efficiency of our framework. They are not targeted for a complete and novel solution, thus most of mechanisms are borrowed from previous work. 

\begin{figure}[!t]
\begin{center}
\includegraphics[width=3.3in]{doubletake/figure/buffer-overflow-diagram}
\end{center}
\caption{Heap organization used to provide evidence of buffer overflow errors. Object headers and unrequested space within allocated objects are filled with canaries; a corrupted canary indicates an overflow occurred.
\label{fig:buffer-overflow}}
\end{figure}

\begin{figure}[!t]
\begin{center}
\includegraphics[width=3.3in]{doubletake/figure/dangling-pointer-diagram}
\end{center}
\caption{Evidence-based detection of dangling pointer (use-after-free) errors. Freed objects are deferred in a quarantine in FIFO order and filled with canaries. A corrupted canary indicates that a write was performed after an object was freed.
\label{fig:dangling-pointer}}
\end{figure}


\subsubsection{Detection of Heap Buffer Overflows}
\label{sec:overflow}
Buffer overflows occurs when a program writes outside of the range of an allocated object. Buffer overflows can greatly affect the reliability and security of a program. 

\paragraph{Detection.}
To detect heap buffer overflows, \DoubleTake{} puts canaries before and after actual heap objects, which is described in Figure~\ref{fig:buffer-overflow}. This method is borrowed from previous work~\cite{overflow:lbc, AddressSanitizer}.
All heap objects of \doubletake{} are managed by power of two size, adding two words for canaries (16 bytes) for each heap object. For an allocation size not power of two size (a non-aligned object), \doubletake{} rounds it up to the next power of two size class, putting byte-based canaries and word-based canaries immediately after this object. This approach lets \doubletake{} catch overflows as small as one byte. 

Detection happens in the following scenario. At memory deallocation, \doubletake{} checks buffer overflows only for non-aligned objects, with size different with the power of two class. At the end of each epoch, \doubletake{} verifies integrities of all canaries by traversing the bitmap, which marks the placement of canaries and discussed in Section~\ref{sec:canaries}.  Corrupted canaries indicates heap buffer overflows. 

\paragraph{Re-execution.}
\label{sec:overflowreport}
When a program is found to have heap overflows, \doubletake{} rolls back the execution to the most recent checkpoint and re-executes this program.
To precisely identifying instructions responsible for a buffer overflow, \doubletake{} installs a watch point at the address of a corrupted canary before re-execution. When the program is re-executed, any instruction that writes to this address will trigger a debug trap (resulting in a SIGTRAP signal). \doubletake{} then reports the exact call site of trapped instructions, acquired by calling \texttt{backtrace} function. To further help programmer, \doubletake{} can also report the allocation site of an overflowed heap object. 
  
\subsubsection{Detection of Memory Leak}
\label{sec:leak}
Heap memory is leaked when it becomes inaccessible without being freed. Memory leak can greatly reduce the performance and availability of programs, which makes it one of common classes of reported bugs.

\paragraph{Detection.}
\doubletake{} detects memory leak using the same marking mechanism as conservative garbage collection ~\cite{Wilson:1992:UGC:645648.664824}. \doubletake{} marks the reachability of all heap objects: whether a heap object is reachable from the globals, the stack, and registers. Any unreachable object that has not been freed must be leaked. 

In order to compute the reachability from globals, stack, and registers, \doubletake{} starts to put all possible pointers, any eight-byte aligned value falling into the range of the heap, into a work queue. Then \doubletake{} checks all values in the queue using a breadth-first search algorithm: for each value, \doubletake{} verifies whether this value is a valid address inside a heap object; If this valid object is still non-reachable (not checked before), \doubletake{} marks it as reachable and also put all possible pointers inside this object into the work queue. After this step, this value is removed from the work queue. \doubletake{} stops when there is no value in the work queue any more. In this time, all heap objects which is reachable should have been marked. 

After the marking phase, \doubletake{} traverses the whole heap to find leaked objects, those un-reachable non-freed objects. To simplify the marking and checking phase, the status of an object, whether it is freed or reachable are marked in the object header. During this traverse, \doubletake{} recovers the status of every object to un-reachable. \doubletake{} also adds all leaked objects into a global hash table, where each leaked object also keeps information of its starting address and size. 

\paragraph{Re-execution.}
\label{sec:leakreport}
\doubletake{} uses re-execution to find allocation sites for all leaked heap objects. Re-execution proceeds as normal, with an added check in each \texttt{malloc()} function call. When a memory allocation matches the actual size and address of any leaked heap object, 
\doubletake{} reports its call stack, obtaining by using \texttt{backtrace} 
functions of \texttt{glibc} library. 

\subsubsection{Detection of Use-after-free Errors}
\label{sec:danglingpointer}
Memory use-after-free errors occur when an application continues to use a pointer after its corresponding object has been deallocated.
Writing to a freed memory can overwrite the contents of lived objects, leading to unexpected behavior.  

\paragraph{Detection.}
To detect memory use-after-free errors, 
\doubletake{} firstly delays memory re-usage of all freed objects 
by putting them into a quarantine list, same as AddressSanitizer does ~\cite{AddressSanitizer}. 
Those objects in the quarantine list are actually returned back
to the program heap when the total size of freed objects in the quarantine list are larger then a pre-defined threshold or the quarantine list is full.

In order to find evidence of memory use-after-free problems, 
\doubletake{} fills the first 128 bytes of an object, which can be adjustable, with canaries. 

Those canaries are checked before an object is returned back to the program heap or in the end of an epoch. Same as detection of buffer overflows, 
a corrupted canary indicates a use-after-free memory error and must be reported to user. 

\paragraph{Re-execution}
When a program has use-after-free memory errors, \doubletake{} re-executes this program to find out the allocation and deallocation site of corresponding objects and those instructions actually writing them.
%It is possible that multiple instructions can access an object after deallocation.  

To obtain an object's call site of allocation and deallocation, 
\doubletake{} checks starting address of each object during every memory allocation and deallocation. If an object has the same address as those use-after-free objects, its corresponding allocation/deallocation site are saved. To find out those instructions writting a freed object, 
\doubletake{} installs hardware watch points on those violating addresses, which shares the same mechanism as overflow detection.
\doubletake{} reports an use-after-free error, with its allocation site and deallocation site, in order to help programmers locate a memory use-after-free error. 